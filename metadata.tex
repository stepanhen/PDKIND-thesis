%%% Please fill in basic information on your thesis, which will be automatically
%%% inserted at the right places.

% Type of your thesis:
%	"bc" for Bachelor's
%	"mgr" for Master's
%	"phd" for PhD
%	"rig" for rigorosum
\def\ThesisType{bc}

% Language of your study programme:
%	"cs" for Czech
%	"en" for English
\def\StudyLanguage{cs}

% Thesis title in English (exactly as in the official assignment)
% (Note: \xxx is a "ToDo label" which makes the unfilled visible. Remove it.)
\def\ThesisTitle{The PD-KIND algorithm in the Golem CHC solver}

% Author of the thesis (you)
\def\ThesisAuthor{Štěpán Henrych}

% Year when the thesis is submitted
\def\YearSubmitted{2024}

% Name of the department or institute, where the work was officially assigned
% (according to the Organizational Structure of MFF UK in English,
% see https://www.mff.cuni.cz/en/faculty/organizational-structure,
% or a full name of a department outside MFF)
\def\Department{Department of Distributed and Dependable Systems}

% Is it a department (katedra), or an institute (ústav)?
\def\DeptType{Department}

% Thesis supervisor: name, surname and titles
\def\Supervisor{doc. RNDr. Jan Kofroň, Ph.D.}

% Supervisor's department (again according to Organizational structure of MFF)
\def\SupervisorsDepartment{Department of Distributed and Dependable Systems}

% Study programme (does not apply to rigorosum theses)
\def\StudyProgramme{Informatika}

% An optional dedication: you can thank whomever you wish (your supervisor,
% consultant, who provided you with tea and pizza, etc.)
\def\Dedication{%
\xxx{Dedication.}
}

% Abstract (recommended length around 80-200 words; this is not a copy of your thesis assignment!)
\def\Abstract{%
PDKind (Property-Directed K-induction) is a combination of IC3 and k-induction, both commonly used model checking algorithms. PDKind separates reachability checking from induction reasoning, allowing induction to be replaced by k-induction. This work focuses on analysing and implementing the PDKind algorithm in the Golem solver as a back-end engine. By integrating PDKind into Golem, our goal is to take advantage of its unique approach to improve the solver's effectiveness in handling various verification tasks. The implementation will be tested and compared to other engines within Golem on a set of benchmarks to demonstrate its comparable efficiency.
}
% 3 to 5 keywords (recommended) separated by \sep
% Keywords are useful for indexing and searching for the theses by topic.
\def\ThesisKeywords{%
PDKind \sep Golem \sep Model checking
}

% If any of your metadata strings contains TeX macros, you need to provide
% a plain-text version for use in XMP metadata embedded in the output PDF file.
% If you are not sure, check the generated thesis.xmpdata file.
\def\ThesisAuthorXMP{\ThesisAuthor}
\def\ThesisTitleXMP{\ThesisTitle}
\def\ThesisKeywordsXMP{\ThesisKeywords}
\def\AbstractXMP{\Abstract}

% If your abstracts are long and do not fit in the infopage, you can make the
% fonts a bit smaller by this setting. (Also, you should try to compress your abstract more.)
\def\InfoPageFont{}
%\def\InfoPageFont{\small}  % uncomment to decrease font size

% If you are studing in a Czech programme, you also need to provide metadata in Czech:
% (in English programmes, this is not used anywhere)

\def\ThesisTitleCS{Algoritmus PD-KIND v řešiči Golem}
\def\DepartmentCS{Katedra distribuovaných a spolehlivých systémů}
\def\DeptTypeCS{Katedra}
\def\SupervisorsDepartmentCS{Katedra distribuovaných a spolehlivých systémů}
\def\StudyProgrammeCS{Informatika}
\def\ThesisKeywordsCS{%
PDKind \sep Golem \sep Model checking
}

\def\AbstractCS{%
PDKind (Property-Directed K-induction) je kombinací IC3 a k-indukce, obou běžně používaných algoritmů pro model checking. PDKind odděluje ověření dosažitelnosti od indukčního odůvodňování, což umožňuje nahradit indukci k-indukcí. V této práci se zaměřujeme na analýzu a implementaci algoritmu PDKind jako back-endovém enginu v řešiči Golem. Integrací PDKind do Golemu chceme využít jeho jedinečný přístup ke zlepšení efektivity řešiče při řešení různých verifikačních úloh. Implementace bude testována a porovnána s ostatními enginy řešiče Golem na sadě benchmarků, aby se prokázala její srovnatelná účinnost.
}
