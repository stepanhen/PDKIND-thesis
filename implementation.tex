\chapter{Implementation}
\noindent In this chapter, we will analyze the PDKind algorithm and describe
how it was adapted to our implementation. We will go into the structure of the
implementation, cover the individual components of the engine, and explain the
decisions made during development. Finally, we will describe how the engine was
integrated into the Golem solver.

To keep the structure consistent with Golem's engine system, we placed our
solution in the files \texttt{PDKind.cpp} and \texttt{PDKind.h}, without
creating separate files for individual components.

\section{Golem structures}
\noindent In this section, we will describe the Golem's data structures and procedures that we utilize in our solution. 

\subsection{PTRef}
\noindent The main data structure used in Golem is \texttt{PTRef} from OpenSMT.
\texttt{PTRef} is a reference structure that points to another structure,
representing a single term in a formula called \texttt{PTerm}. \texttt{PTRef}
is just a number as an identifier to differentiate between the terms. The
mapping between \texttt{PTRef} and \texttt{PTerm} is handled by the
\texttt{Logic} class, respectively, one of its implementations. The
\texttt{Logic} class keeps a mapping table between the \texttt{PTRef} and
\texttt{PTerm}. It also provides methods that we can use to create new terms.
Each implementation of the \texttt{Logic} class also has functions according to
the theory it uses. In our solution, we use the QF\_LRA theory. Therefore, we
will be using this theory in the OpenSMT too. 

We don't need to delve into much more detail here because, in our solution, we
will be directly using only \texttt{PTRef} and some basic functions of the
\texttt{Logic} class, shown in Figure \ref{ex:PTref}.

\renewcommand{\figurename}{Figure}
\lstset{style=cppstyle}
\begin{figure}[h]
\begin{lstlisting}
PTRef x = logic.mkIntVar("x");
PTRef y = logic.mkIntVar("y");

PTRef formula = logic.mkAnd(
    logic.mkEq(x, logic.getTerm_IntOne),
    logic.mkEq(y, (logic.mkPlus(x, logic.getTerm_IntOne))));
\end{lstlisting}
\caption{PTRef and Logic usage example}\label{ex:PTref}
\end{figure}
The first two lines initiate \texttt{x} and \texttt{y} as an integer variable.
On the fourth line, we use the \texttt{Logic} class to create formulas $(x =
1)$ and $(y = x + 1)$, and then we use it again to get a conjunction of these
formulas and store it in the \texttt{PTRef formula}.

\subsection{MainSolver}
\noindent Another structure that we will use is the \texttt{MainSolver} class
from OpenSMT. \texttt{MainSolver} will serve us as the main solver for
satisfiability checking, model generation, and interpolation. The solver is
initialized with a config.

In the config, we can, for example, specify, if we need to generate
interpolants because the solver only produces models by default. In Figure
\ref{ex:MainSolver}, we can see an example
of the \texttt{MainSolver} usage with model generation and interpolation.

\begin{figure}[h]
\begin{lstlisting}
SMTConfig config;
config.setOption(SMTConfig::o_produce_inter, SMTOption(true), "ok");
config.setSimplifyInterpolant(4);

MainSolver solver (logic, config, "Example solver");
solver.insertFormula(A);
solver.insertFormula(B);
solver.insertFormula(C);

sstat result = solver.check();
if(result == s_True) {
    std::unique_ptr<Model> model = solver.getModel();
    // Do something with the model...
} else {
    auto itpContext = solver.getInterpolationContext();
    vec<PTRef> itps;
    int mask = 3;
    itpContext->getSingleInterpolant(itps, mask);
    assert(itps.size() == 1);
    PTRef interpolant = itps[0];
    // Do something with the interpolant...
}
\end{lstlisting}
\caption{MainSolver usage example}\label{ex:MainSolver}
\end{figure}
On lines 1-3, we initialize the config for the solver. The function on line
number 3 chooses an interpolating algorithm with the value 4. On lines 5-8, we
initiate the solver and insert three formulas, A, B, and C. Lines 10-13 are
pretty simple. We call the \texttt{check()} procedure, and if the result is
\texttt{s\_True}, we get the model.

Otherwise, we can get the interpolant. The crucial part is choosing the correct
mask. The mask represents which formulas in the solver should be included in
the interpolation. In the binary representation of the mask, the i-th bit
corresponds to an i-th formula in the solver frame. A bit of value 1 includes
the corresponding formula, and 0 doesn't. In our case, the mask is equal to 3
(in binary 011). Therefore, we will create an interpolant of $A \wedge B$
because we include A and B and exclude C.

\section{PDKind structures}
\noindent In this section, we analyze and describe the data structures and procedures implemented in our engine satisfying the methods and definitions shown in chapter \ref{PD-Kind}. Also we describe how they interact with the OpenSMT solver, as shown in figure \ref{fig:pdkind_diagram}.


\begin{figure}[H]
    \centering
\begin{tikzpicture}[node distance=1cm]

% Nodes% Define styles for the large and small rectangles
\tikzstyle{largest}=[rectangle, draw, minimum width=6cm, minimum height=4cm]
\tikzstyle{large}=[rectangle, draw, minimum width=5cm, minimum height=4cm]
\tikzstyle{small}=[rectangle, draw, minimum width=1.5cm, minimum height=1cm]

% Draw the large rectangles
\node[large, label={above:PDKind}] (PDKind) at (0, 0) {};
\node[largest, label={above:ReachabilityChecker}] (ReachabilityChecker) at (7, 0) {};

% Draw the small rectangles inside PDKind
\node[small] (PDK1) at (-1, -0.75) {push};
\node[small] (PDK2) at (1, -0.75) {solve};

% Draw the small rectangles inside ReachabilityChecker
\node[small] (RC1) at (5.75, -0.75) {generalize};
\node[small] (RC2) at (8.25, -0.75) {reachable};
\node[small] (RC3) at (7,1) {checkReachability};

\node[decision] (OpenSMT) at (3.25,-4) {OpenSMT};


\draw [thick,->,>=stealth, color=blue] (8.25, -1.25) |- (8.25, -4) -- (OpenSMT);
\draw [thick,->,>=stealth, color=blue] (5.75, -1.25) |- (5.75, -4) -- (OpenSMT);


\draw [thick,->,>=stealth, color=blue] (RC2) -- (RC1) ;
\draw [thick,->,>=stealth, color=blue] (-1, -0.25) |- (-1, 1) -- (RC3);
\draw [thick,->,>=stealth, color=blue] (-1, -0.25) |- (-1, 1) -- (4.375, 1) -| (4.375, -0.75) -- (4.75, -0.75);
\draw [thick,->,>=stealth, color=blue] (RC3) -- (9.625,1) -| (9.625, -0.75) -- (9.25, -0.75) ;

\draw [thick,->,>=stealth, color=blue] (PDK1) |- (-1, -4) -- (OpenSMT);

\draw [thick,->,>=stealth, color=blue] (1.75, -0.75) |- (3.25, -0.75) -- (OpenSMT);
\draw [thick,->,>=stealth, color=blue] (PDK2) -- (PDK1) ;

\end{tikzpicture}
\caption{Architecture of PDKind engine}
    \label{fig:pdkind_diagram}
\end{figure}

\subsection{ReachabilityChecker class}

The reachability procedure, shown in algorithm \ref{alg:1}, determines whether
a given state is reachable within a specific number of steps from the initial
states. This check is performed using a depth-first search (DFS) strategy,
along with several supporting methods and data structures that are crucial for
the correctness and efficiency of our implementation.

In particular, we analyze the following components: \textbf{Satisfiability checking}, \textbf{Generalization}, \textbf{Explanation} and \textbf{Reachability frames} representation.
\subsubsection*{Satisfiability Checking}

\noindent For satisfiability checking, we could choose from various SMT
solvers, including Z3\cite{10.1007/978-3-540-78800-3_24},
cvc5\cite{10.1007/978-3-030-99524-9_24}, or
OpenSMT\cite{10.1007/978-3-642-12002-2_12}, each offering different tradeoffs
in terms of theory support and performance.

Since Golem already uses OpenSMT in its existing engines and provides a wrapper
around it, we chose to use it in our implementation as well. This ensures
consistency across the project and avoids the need to integrate a new solver.
More importantly, using the same solver allows for a fair performance
comparison between PDKind and other Golem engines, as any observed differences
are then due to algorithmic behavior rather than backend solver performance.

OpenSMT also provides two essential features for our implementation: model
generation and the ability to compute interpolants. The latter is particularly
important for strengthening inductive invariants. Leveraging these built-in
features allows us to avoid implementing additional mechanisms, making OpenSMT
the most convenient and efficient choice for our engine.

\subsubsection*{Generalization}

\noindent Generalization plays an essential role in PDKind by preventing
redundant checks and improving abstraction. It allows the algorithm to reason
about sets of states rather than individual states. However, Golem does not
provide a built-in generalization method.

Fortunately, it offers the components needed to construct our own method, as
shown in algorithm \ref{alg:2}. The generalization
is performed by removing non-state variables from a formula using the
\texttt{KeepOnly} function. Given a formula of the form \( T \wedge C \), where
\( T \) is a transition formula and \( C \) is a state formula representing a
counterexample, we extract only the parts over state variables \( \vec{x} \).
This yields a generalized formula \( G \), which satisfies the generalization
conditions in definition \ref{generalization}.

\begin{figure}[H]
    \begin{mdframed}
        \begin{algorithmic}[1]
            \State \textbf{Input:} Model $M$, transition formula $T$, state formula $C$
            \State \textbf{Output:} Generalized formula $G$
            \State $StateVars \gets$ GetStateVars()
            \State $G \gets$ KeepOnly(StateVars, $T \wedge C$, $M$)
            \State \Return $G$
        \end{algorithmic}
    \end{mdframed}
    \caption{Generalize method}\label{alg:2}
\end{figure}

\paragraph{Notation.}
\noindent In algorithm \ref{alg:2}, we work with the following inputs:
\begin{itemize}
    \item \textbf{$T$ (Transition formula)}: Defines the system's state evolution.
    \item \textbf{$C$ (Counterexample formula)}: Represents a bad state discovered by the algorithm.
    \item \textbf{$M$ (Model)}: A concrete assignment that satisfies \( T \wedge C \) and shows how the system could reach the bad state.
\end{itemize}

From the model \( M \), we construct a formula \( G \) that over-approximates
it while being expressed only over state variables. This serves as a
generalized lemma that can be further refined or blocked.

\noindent The implementation of this method uses Golem's
\texttt{ModelBasedProjection} utility to perform the variable elimination and
projection.

Although the \texttt{generalize()} method is used by both the
\texttt{reachable()} function and the \texttt{push()} procedure, we placed it
inside the \texttt{ReachabilityChecker} class. Since this class is already
responsible for generating interpolants (i.e., explanations), it was a natural
place to handle generalizations as well.

\vspace{\baselineskip}
\noindent\textbf{KeepOnly method}

\vspace{\baselineskip}
\noindent The \texttt{KeepOnly} method extracts only the state variables from
the formula \( T \land C \) by removing all auxiliary variables, specifically
the intermediate states \( \vec{w} \) and successor states \( \vec{y} \), as
described in definition \ref{generalization}. This ensures that the resulting
generalized formula \( G(\vec{x}) \) depends only on the current state
variables. As a result, \( G \) becomes an over-approximation: it captures not
just the specific assignment given by model \( M \), but all current states
that satisfy \( T \land C \).

\newpage
\noindent\textbf{Satisfying the Generalization Definition}

\vspace{\baselineskip}
\noindent To confirm that algorithm \ref{alg:2}
satisfies the generalization definition, we verify
the two required properties:

\begin{enumerate}
    \item \textbf{Over-approximation:}
    \begin{itemize}
        \item By removing \( \vec{w} \) and \( \vec{y} \), we ensure that \( G(\vec{x}) \) abstracts all possible satisfying assignments of the transition relation \( T \) and condition \( C \).
        \item This directly satisfies the first requirement in Definition~\ref{generalization}.
    \end{itemize}

    \item \textbf{Consistency with Initial States \( A(\vec{x}) \):}
    \begin{itemize}
        \item Although \( A \) is not directly present in the generalization formula, definition \ref{generalization} requires that the generalization remains consistent with it.
        \item Since we only remove variables and do not add constraints, \( G(\vec{x}) \) does not exclude any valid states reachable from the initial condition \( A(\vec{x}) \).
        \item This guarantees that \( G(\vec{x}) \) does not eliminate any valid initial-state successors and satisfies the second requirement in the definition.
    \end{itemize}
\end{enumerate}

\noindent Thus, algorithm \ref{alg:2} produces a valid and useful generalization method for PDKind.

\subsubsection*{Explanation} \label{Explain}
\noindent Instead of implementing a custom \( Explain() \) method, we utilize
the OpenSMT solver feature, which is an interpolation. To obtain the
interpolant, we tell the solver to compute an interpolation of the first two
formulas inserted in it. For example, given a satisfiability check $CheckSat(A
\land B \land C)$, we want an interpolant of $A \land B$.

\begin{definition}\textbf{Craig Interpolation \cite{7fb35eae-c1f2-3be3-bbd7-8ec25936313a}:}
Let \( A \) and \( A' \) be first-order formulas such that \( A \models A' \).  
If \( A \) and \( A' \) share at least one predicate symbol, then there exists a formula \( I \) (called an interpolant) such that:
\begin{itemize}
    \item \( A \models I \)
    \item \( I \models A' \)
    \item All predicate symbols in \( I \) occur in both \( A \) and \( A' \)
\end{itemize}
\end{definition}

\noindent The use of interpolation as an explanation step is directly supported by the
design of PDKind. As described in \cite{7886665}, interpolants are used to
block infeasible counterexamples in a way that preserves soundness. The
interpolant overapproximates the set of states from which a violation could
occur but is still inconsistent with the property being violated. In our
implementation, when a spurious counterexample is detected, the interpolant we
extract serves as a weakening of the generalization that excludes the
infeasible trace. Because the interpolant is implied by the reachable prefix
and inconsistent with the bad state, it blocks the counterexample while
preserving states that are still potentially valid. This makes it a suitable
and valid explanation in the PDKind algorithm, consistent with the original
paper.

On line 13 in Algorithm \ref{alg:1}, we require an
interpolant from the \( CheckSAT() \) call that occurred in the previous \(
Reachable() \) call. To support this behavior, we modify our \( Reachable() \)
method on line 17 to return an interpolant along with the \( false \) result
when a check fails. Since interpolation is only defined when the formula is
unsatisfiable, we only request an interpolant if \( CheckSAT() \) returns \(
false \).

In our implementation, each \( CheckSAT() \) is performed using a separate
solver instance. This means obtaining an interpolant is simple; we instruct the
solver instance to return the interpolant generated from its internal state,
reflecting the formulas used in the failed check.

\subsubsection*{Reachability frames}
\noindent For the \( Reachability frames \) representation, we had several
choices. The simplest approach was to create a list $R$ of formulas, where each
frame is stored as $R_i := R[i]$. Each time we call \( Reachable() \) from
outside, we would construct a new list of reachability frames.

While this method is straightforward, it is inefficient because it discards
previously computed frames with each call, preventing reuse and requiring
redundant computation. 

To address this, we implement a \( Reachability \) class that maintains a
persistent list of frames. Each call to \( Reachable() \) extends the existing
list rather than creating a new one. This way, we can reuse previously computed
information across multiple calls and avoid redundant recomputation.

As mentioned in the paper \cite{7886665}, we need a structure that holds
formulas representing the set of $k$-reachable states from the initial states,
for all $k \in \{1, \dots, n\}$. We also need to access and update these
formulas by index. For this purpose, we define the structure \texttt{RFrames},
which maintains a vector \texttt{r} of formulas \texttt{PTRef}, where $r[i]$ is
the over-approximation of states that can reach the target in exactly $k - i$
steps, or equivalently, states from which the target is reachable in $k - i$
transitions.

Inserting a formula \texttt{f} is creating a conjunction of \texttt{f} and
\texttt{r[i]} and storing it in \texttt{r[i]}. As shown in Figure
\ref{code:RFrames}, we overloaded the \texttt{[]}
operator to allow quick access to the vector. Also, if \texttt{r[i]} doesn't
exist, we fill the vector from \texttt{r.size()} up to \texttt{i} with the term
\texttt{true}. Before updating the i-th frame, the insert method fills up the
vector too, if \texttt{r[i]} doesn't exist.

\begin{figure}[H]
\begin{lstlisting}
class RFrames {
    std::vector<PTRef> r;
    Logic & logic;
public:
    RFrames(Logic & logic) : logic(logic) {}

    PTRef operator[] (size_t i) {
        if (i >= r.size()) {
            while (r.size() <= i) {
                r.push_back(logic.getTerm_true());
            }
        }
        return r[i];
    }

    void insert(PTRef fla, size_t k) {
        if (k >= r.size()) {
            while (r.size() <= k) {
                r.push_back(logic.getTerm_true());
            }
        }
        PTRef new_fla = logic.mkAnd(r[k], fla);
        r[k] = new_fla;
    }
};
\end{lstlisting}
\caption{RFrames structure}\label{code:RFrames}
\end{figure}

\subsubsection*{Implementation}
\noindent Now, we can move on to the actual implementation of the \texttt{ReachabilityChecker} class, which brings together the components discussed in the previous sections.

The \texttt{ReachabilityChecker} class encapsulates the mechanism needed for performing reachability checks.
As analyzed earlier, creating a new instance of the \texttt{RFrames} structure
for each reachability check would be inefficient. Therefore, we maintain a
single instance of \texttt{RFrames} as a member of the
\texttt{ReachabilityChecker} class and reuse it across all reachability
queries.

Although the \texttt{ReachabilityChecker} class could be placed in its own file, we follow the structure of Golem and include it in the engine file alongside the rest of the code.

\vspace{\baselineskip}
\noindent\textbf{Reachable function:}

\vspace{\baselineskip}
\noindent The \texttt{reachable(unsigned k, PTRef formula)} function is the core of the
\texttt{ReachabilityChecker} class. It implements the pseudocode shown in
algorithm \ref{alg:1}, but for the function to
integrate correctly with the rest of the engine, several modifications are
necessary.

First, we must return a formula alongside the reachability result. If the
\texttt{reachable()} function returns \texttt{false}, we ask the solver to
produce an interpolant and return it together with the result. Otherwise,
we return \texttt{True} and a \texttt{false} term.

To generate interpolants, we need to set up a solver config at the beginning of
the function. This config is then used to initialize solvers for both
satisfiability checking and interpolant generation.

The first solver is used when \texttt{k = 0}, to check whether the given
formula holds in the initial states. If this check returns \texttt{false}, we
proceed to generate the interpolant as shown in
figure \ref{ex:MainSolver}, using
\texttt{mask = 1}. This is because we do not insert the transition formula into
the solver, so we effectively compute an interpolant for $A \land B$ with
respect to $A$.

If the given \texttt{k} is greater than zero, we proceed with the while loop.
Before doing so, we must shift the version of the input \texttt{formula} from 0
to 1. We do this using the \texttt{TimeMachine} and its
\texttt{sendFlaThroughTime(PTRef fla, int steps)} function, which increments
the formula version by the given number of steps.

Versioning is crucial here because we are verifying whether the
\texttt{formula} can be reached via a single transition from a specific RFrame.
Therefore, formulas added to the solver must follow the structure:  
\[
(R[k{-}1]_0 \wedge T_{0,1} \wedge formula_1)
\]  
As we will observe, each \texttt{R[i]} is consistently versioned to 0.

In section \ref{Explain}, we analyzed that the
\texttt{Explain()} method on line 13 in algorithm \ref{alg:1}
is omitted, and instead, the interpolant is returned
by the \texttt{Reachable()} function, which is called on line 10.

The pseudocode currently concludes after these steps, but an additional task
remains: generating the interpolant when \texttt{k > 0}. In the while loop, if
the satisfiability check fails, we request an interpolant from the solver in
the same way as shown in
figure \ref{ex:MainSolver}. However, the
interpolant must be shifted back by one version using the \texttt{TimeMachine},
ensuring it has version 0. This aligns with our earlier note that all
\texttt{R[i]} are versioned to 0, and this interpolant will be used to update
them.

Yet, this interpolant alone is not sufficient. We must also verify whether the
given \texttt{formula} holds in the initial states. If it does, we return the
interpolant. If it does not, we create another interpolant for the initial
state check as we did in the \texttt{k = 0} case, and return the disjunction of
the two interpolants.

\textbf{CheckReachability function:}
\noindent In the Push procedure, we need to check whether a formula is
reachable over a range of steps, rather than a fixed number. To support this,
we define an extended version of the \texttt{reachable()} function, which takes
a step range \texttt{(k\_from, k\_to)} and iterates through the values using a
loop. For each \( k \), we call \texttt{reachable(k, formula)} and return the
result as soon as we encounter a successful check. If none of the calls
succeed, the function returns the result of the last failed call, including the
number of steps used and the interpolant generated by the last solver instance.
This interpolant serves as an explanation of why the formula is unreachable.

We initialize a new solver instance for each reachability check. Although this
introduces some performance overhead, it is necessary to keep model and
interpolant generation isolated, avoiding side effects from previous solver
states. This approach ensures correctness, especially when extracting models
and interpolants using OpenSMT, which does not return models by default unless
configured. It also allows us to efficiently find the earliest reachable step,
improving precision in the overall algorithm.

\subsection{Push procedure}

\noindent The \texttt{push()} procedure, which implements
algorithm \ref{alg:3}, is the core of our solution. It
connects all parts of our engine to either produce inductive strengthening or
find a counterexample.

\subsubsection{Induction Frame}

\noindent The induction frame is a key data structure in the PDKind algorithm.
It is used to store lemmas that lead to constructing inductive invariants. The
induction frame is also discussed in the paper \cite{7886665} and formally
defined in section \ref{def:IFrame}.

In our implementation, we represent the \( Induction\ Frame \) as a set of objects, where each object consists of:
\begin{itemize}
    \item A lemma, representing an inductive candidate formula.
    \item A counterexample structure, which stores additional information about the counterexample.
\end{itemize}

Each \texttt{IFrameElement} holds a \texttt{PTRef lemma} and a
\texttt{CounterExample counter\_example}. We use the dedicated
\texttt{CounterExample} structure instead of a plain \texttt{PTRef} because it
allows us to store both the formula and the number of steps required to reach
it from the initial states. This additional data is crucial for generating
unsatisfiability witnesses when proving that a property is \( \mathsf{UNSAFE}
\).

In our solution, we implement the frame as a set of \texttt{IFrameElement}
instances stored in the \texttt{InductionFrame} structure.

\subsubsection{Transition Construction}

\noindent In the code snippet in
Figure~\ref{code:TransitionGen}, we show
how we created the transition $T[F_{ABS}]^k$, which satisfies the definition
shown in Section~\ref{ReachabilityChecking}.

We utilize the \texttt{TimeMachine} and \texttt{Logic} structures to generate a
conjunction of transitions $t_{0,1} \wedge t_{1,2} \wedge \dots \wedge
t_{k-1,k}$, with versions ranging from 0 to \( k-1 \). Additionally, we create
a conjunction of formulas $f\_abs_0 \wedge f\_abs_1 \wedge \dots \wedge
f\_abs_{k-1}$, also versioned from 0 to \( k-1 \). Finally, we take a
conjunction of these two results to obtain the final transition formula.

\begin{figure}[H]
\begin{lstlisting}
PTRef t_k = transition;
PTRef f_abs_conj = logic.getTerm_true();
std::size_t i;
for (i = 1; i < k; ++i) {
    PTRef versionedFla = tm.sendFlaThroughTime(iframe_abs, i);
    PTRef versionedTransition = tm.sendFlaThroughTime(transition, i);
    t_k = logic.mkAnd(t_k, versionedTransition);
    f_abs_conj = logic.mkAnd(f_abs_conj, versionedFla);
}

PTRef t_k_constr = logic.mkAnd(t_k, f_abs_conj);
\end{lstlisting}
\caption{Transition initialization}\label{code:TransitionGen}
\end{figure}

\subsubsection{Return Values and Result Tracking}

\noindent The rest of the implementation follows the pseudocode, but we added
some parts to enable the production of witnesses.

First, we needed to extend the return values by an additional value, which
represents the number of steps to a counterexample. The \texttt{push()}
procedure now returns five different values. For clarity, we encapsulated these
values in a structure called \texttt{PushResult}, as shown in
figure \ref{code:PushResult}.

\begin{figure}[H]
\begin{lstlisting}
struct PushResult {
    InductionFrame i_frame;
    InductionFrame new_i_frame;
    int n;
    bool is_invalid;
    int steps_to_ctx;
    PushResult(InductionFrame i_frame,
               InductionFrame new_i_frame,
               int n,
               bool is_invalid,
               int steps_to_ctx) { ... }
};
\end{lstlisting}
\caption{PushResult structure}\label{code:PushResult}
\end{figure}

\subsubsection{Witness Construction}

\noindent In the next step, we must update the \texttt{steps\_to\_ctx} value,
which was set to 0 during initialization. This update occurs in the same part
of the code where we set the \texttt{isInvalid} flag to \texttt{True}. The new
value will be the sum of the number of steps required to reach the
counterexample from \texttt{g\_cex} (i.e., \texttt{g\_cex.num\_of\_steps}) and
the value \texttt{k} returned by the \texttt{CheckReachability()} function.
This tells us that \texttt{g\_cex} is \( k \)-reachable from the initial
states, so the final value of \texttt{steps\_to\_ctx} will be:  
\[
\texttt{g\_cex.num\_of\_steps + k}
\]

Finally, we need to assign the \texttt{num\_of\_steps} value for each newly
created counterexample. We already analyzed in section \ref{UNSATWit}
that a new counterexample is only generated once.
This process is shown in figure \ref{code:CEX}.

The new counterexample is created by generalizing the previous potential
counterexample and increasing its version by \texttt{k}. Therefore, the value
of \texttt{num\_of\_steps} will be the same as the previous counterexample's
\texttt{num\_of\_steps}, incremented by \texttt{k}.

\begin{figure}[H]
\begin{lstlisting}
PTRef f_cex = tm.sendFlaThroughTime(obligation.counter_example.ctx, i);
MainSolver solver2(logic, config, "f_cex reachability");
solver2.insertFormula(iframe_abs);
solver2.insertFormula(t_k_constr);
solver2.insertFormula(f_cex);
auto res2 = solver2.check();
if (res2 == s_True) {
    auto model2 = solver2.getModel();
    CounterExample g_cex(reachability_checker.generalize(*model2, t_k, f_cex), obligation.counter_example.num_of_steps + k);
\end{lstlisting}
\caption{CounterExample initialization}\label{code:CEX}
\end{figure}

\subsubsection{Model Generation}

\noindent At last, we make a minor adjustment. In the pseudocode, the model is
obtained directly from the \texttt{CheckSAT()} command. However, in our
implementation, we request the model only after verifying that the result of
\texttt{CheckSAT()} was positive. Therefore, the call to extract the model is
placed after an \texttt{if} statement checking the satisfiability result.

\subsection{PDKind Engine}
\noindent In this section, we describe how we implemented the PDKind engine and
integrated it into the Golem solver. The engine ties together all parts of the
algorithm and runs the main solving loop.

The \texttt{PDKind} class inherits from the \texttt{Engine} class in
Golem, which defines a virtual method \texttt{solve(ChcDirectedHyperGraph const
\& graph)}. In our implementation, we override this method to solve the
transition system using the PDKind algorithm.

This method accepts a hypergraph that represents the verification task. First,
we convert the hypergraph into a normal graph. To solve the normal graph, we
overload the method \texttt{solve(ChcDirectedGraph const \& system)}, which
operates on this converted structure.

Within the overloaded method, we check whether the graph is trivial. If so, we
use Golem’s \texttt{Common} class and its \texttt{solveTrivial()} function to
handle it. If the graph is non-trivial, we transform it into a
\texttt{TransitionSystem} and pass it to
\texttt{solveTransitionSystem(TransitionSystem const \& system)}, which runs
the PDKind algorithm.

\subsubsection{SolveTransitionSystem method}

\noindent This function receives a \texttt{TransitionSystem} as input, giving
us access to the initial states, transition relation, and the query, which
represents the set of bad states. The goal is to construct either an inductive
invariant that proves safety or a counterexample that demonstrates unsafety.
The property we attempt to prove is the negation of the query.

This function implements the pseudocode shown in algorithm \ref{alg:4}. Before entering the main loop, we perform two preliminary checks using the \texttt{MainSolver}:
\begin{enumerate}
    \item \textbf{Are the initial states empty?} If the set of initial states is empty, the system is trivially safe and we return \( SAFE \).
    \item \textbf{Does the query already hold in the initial states?} If yes, the system is immediately unsafe and we return \( UNSAFE \).
\end{enumerate}

If neither of these cases applies, we proceed with the main loop of the PDKind
algorithm, which repeatedly calls the \texttt{push()} procedure to search for a
\( k \)-inductive strengthening of the property \( P \).

In our implementation, the \texttt{push()} procedure also provides additional
information needed for generating witnesses. For satisfiability witnesses, we
use the final \texttt{InductionFrame} produced by \texttt{push()} and first
convert it into a \( k \)-inductive invariant. We then use Golem's
\texttt{kinductiveToInductive()} function to transform it into an inductive
invariant.

For unsatisfiability witnesses, we use the \texttt{steps\_to\_ctx} value
returned by \texttt{push()} and store it in the
\texttt{TransitionSystemVerificationResult}, which is then returned by the
engine.

The method terminates when it either constructs a valid \( k \)-inductive
invariant (and returns \( SAFE \)) or discovers a counterexample trace (and
returns \( UNSAFE \)).

\subsection{Validity Checking}

\noindent In many cases, it is often required to provide a witness to the
answer obtained from solving the CHC satisfiability problem. In software
verification, a satisfiability witness corresponds to a program invariant,
while an unsatisfiability witness corresponds to counterexample paths.

Generally:
\begin{itemize}
    \item A \textbf{satisfiability witness} is a model that provides an interpretation of all CHC predicates and variables that satisfy all clauses.
    \item An \textbf{unsatisfiability witness} is a proof presented as a sequence of derivations of ground instances of the predicates. 
\end{itemize}

Witnesses are essential in formal verification, as they provide concrete
evidence to support the solver's conclusion, ensuring that verification results
are both explainable and reproducible.

In Golem \cite{blicha_golem_2023}, each engine provides a validity witness when
the option \texttt{--print-witness} is used. To follow this structure, we
implemented such functionality in the PDKind engine as well.

\subsubsection*{UNSAT Witness} \label{UNSATWit}

\noindent First, we describe the implementation of the unsatisfiability witness
in our engine. The goal is to generate counterexample paths during the CHC
solving process. These paths demonstrate that the safety property does not
hold.

In the PDKind procedure, when a counterexample is detected, we track how many
steps are required to reach it. Rather than storing entire traces, we only
store the number of steps to the counterexample from the initial states. This
keeps the representation compact while still allowing us to reconstruct the
path when needed.

We encapsulate this data in a structure called \texttt{CounterExample}, which includes:
\begin{itemize}
    \item the formula representing the counterexample
    \item the number of steps to reach it from the initial states
\end{itemize}

This information is attached to each tuple in the \( Induction\ Frame \) as \(
(lemma, \) \\ \( counterExample) \). The lemma is valid up to a certain depth, and the
counter-\\example it fails to refute becomes the next candidate in the push
process.

In algorithm \ref{alg:3}, a new counterexample $g_2$
is generated at line 18 and reused in the else branch starting at line 23. The
number of steps assigned to $g_2$ is computed by taking the number of steps
associated with the previous counterexample and adding \( k \), which is the
depth of the current reachability check.

To make this work in the implementation, the \texttt{Push()} procedure was
extended to return an additional field: the total number of steps to the final
counterexample. On line 21, when a reachable counterexample is found, this
number is returned as part of the witness structure.

Finally, the \texttt{solveTransitionSystem()} function records this value in
the \texttt{TransitionSystemVerificationResult}, allowing Golem to generate a
full counterexample trace using built-in utilities when the witness is
requested.

\subsubsection*{SAT Witness}

\noindent Now we describe the construction of a satisfiability witness in the
PDKind engine. Here, the goal is to produce a valid inductive invariant that
proves the safety property.

As described in algorithm \ref{alg:3}, the \( Push()
\) procedure constructs an \texttt{Induction\ Frame}, a set of tuples \( (lemma,
counterExample) \). The lemmas in this frame are valid up to \( n \) steps and
jointly refute all known counterexamples. When the frame stabilizes (i.e., \( F
= G \) at line 11), no new lemmas are needed, and we have a \( k \)-inductive
invariant.

At this point, the PDKind algorithm concludes by returning a \texttt{SAFE}
result (line 12 in algorithm \ref{alg:4}).

To construct the witness, we form a conjunction of all lemmas in the final \(
Induction\ Frame \). This yields an \( n \)-inductive invariant that proves the
property for \( n \) steps. However, to obtain a classical inductive invariant
(valid at all depths), we pass this result to Golem's helper function
\texttt{kinductiveToInductive()}, which transforms the \( n \)-inductive
invariant into a standard inductive invariant.

This invariant is then returned as part of the
\texttt{TransitionSystemVerificationResult} and printed when the user requests
the witness.

\subsubsection*{Summary}

\noindent Validity witnesses not only validate the result of the solver but also increase the trustworthiness and usefulness of the tool. In PDKind:
\begin{itemize}
    \item UNSAT witnesses are derived from tracked steps to a counterexample and reconstructed using Golem's trace utilities.
    \item SAT witnesses are built from the final stable induction frame and converted to inductive invariants.
\end{itemize}

This integration ensures the PDKind engine aligns with the broader structure and expectations of the Golem solver ecosystem.
