\chapter{Golem}

Golem \cite{blicha_golem_2023} is a flexible and efficient solver for CHC satisfiability problems over linear real arithmetic (LRA) and linear integer arithmetic (LIA), written in C++.

Golem integrates an interpolating SMT solver OpenSMT\cite{10.1007/978-3-319-40970-2_35}, and currently implements six different back-end engines for CHC solving, where each engine can use the OpenSMT not only for SMT queries but also for interpolant computation.

\section{Solver overview}

\begin{figure}[ht]
    \centering
\begin{tikzpicture}[node distance=1.5cm]

% Nodes
\node (smt2) [io] {.smt2};
\node (interpreter) [process, right of=smt2, xshift=1.2cm] {Interpreter};
\node (preprocessor) [process, right of=interpreter, xshift=1.2cm] {Preprocessor};
\node (engines) [process, right of=preprocessor, xshift=1.2cm] {Engines};
\node (opensmt) [decision, right of=engines, xshift=1.5cm] {OpenSMT};
\node (result) [result, below of=interpreter] {Result};
\node (sat) [below of=preprocessor, xshift=0.5cm, yshift=0.5cm, font=\footnotesize] {\textcolor{OliveGreen}{SAT + Model}};
\node (unsat) [below of=preprocessor, xshift=0.7cm, yshift=-0.5cm, font=\footnotesize] {\textcolor{red}{UNSAT + Proof}};
% Arrows
\draw [arrow] (smt2) -- (interpreter);
\draw [arrow] (interpreter) -- (preprocessor);
\draw [arrow] (preprocessor) -- (engines);
\draw [arrow] (engines) -- (opensmt);
\draw [arrow] (opensmt) -- (engines);
\draw [arrow, color=OliveGreen] (result) -- (sat);
\draw [arrow, color=red] (result) -- (unsat);

\draw [thick,->,>=stealth, color=blue] (engines.north) to [out=90,in=90] (preprocessor.north);
\draw [thick,->,>=stealth, color=blue] (preprocessor.north) to [out=90,in=90] (interpreter.north);
\draw [thick,->,>=stealth, color=blue] (interpreter) -- (result);
\end{tikzpicture}
\caption{Architecture of Golem}
    \label{fig:golem_diagram}
\end{figure}


\noindent In this section, we will describe the Golem solving process depicted in Figure \ref{fig:golem_diagram}.
\section*{Reading and interpreting CHCs}
\noindent Golem reads the input from a file in the .smt2 format, which is an extension of the SMT-LIB language\cite{BarFT-SMTLIB}. The interpreter builds an internal representation of the CHC system by first normalizing the CHCs to ensure that each predicate has only variables as arguments and then converting the CHCs to the graph representation. The graph representation is then passed to the preprocessor.

\section*{Preprocessing}
\noindent The Preprocessor applies transformations to simplify the graph representation:
        \begin{itemize}
            \item \textbf{Predicate Elimination:} Removes predicates, that are not present in both the body and the head of the same clause.
            \item \textbf{Clause Merging:} Merges clauses with the same uninterpreted predicate by disjoining their constraints.
            \item \textbf{Redundant Clause Deletion:} Removes clauses, that cannot participate in the proof of unsatisfiability. 
        \end{itemize}
    \section* {Engines:}
    \noindent The graph is then solved with one of the engines (this option is specified by the user):
        \begin{itemize}
            \item Bounded Model Checking (BMC) \cite{10.1007/3-540-49059-0_14}
            \item k-Induction (KIND) \cite{10.1007/3-540-40922-X_8}
            \item Interpolation-based Model Checking (IMC) \cite{10.1007/978-3-540-45069-6_1}
            \item Lazy Abstractions with Interpolants (LAWI) \cite{10.1007/11817963_14}
            \item Spacer \cite{10.1007/978-3-319-08867-9_2}
            \item Transition Power Abstraction (TPA) \cite{blicha_golem_2023}
        \end{itemize}

The user can also select the option to produce a validity witness. When the engine solves the problem, it generates a model for the SAT result or a proof for the UNSAT result. These models and proofs are translated back by the preprocessor to match the original system.

\begin{figure}[H]
\begin{align*}
    x \leq 1 &\implies I(x) \quad &\text{\color{red}{\(I(1)\)}} \\
    x' = x + 1 &\implies T(x, x') \quad &\text{\color{red}{\(T(1, 2)\)}} \\
    I(x) \land T(x, x') &\implies S(x') \quad &\text{\color{red}{\(S(2)\)}} \\
    S(x) \land x \geq 2 &\implies \text{\color{red}{false}} \quad &\text{\color{red}{false}}
\end{align*}
    \caption{UNSAT + Proof example}\label{ex:UNSAT}
\end{figure}

In Figure \ref{ex:UNSAT}, we can see a CHC
system and a proof of its unsatisfiability. There are four derivation steps. In
the first step, the unsatisfiability proof shows that the variable $x$ is set
to $1$, giving us $I(1) \impliedby True$. In the second step, we continue with
the initial value of $x$ and proceed to get $x' := 2$ which gives us $T(1, 2)
\impliedby True$. Step three applies resolution to the instance of the third
clause for $x := 1$ and $x' := 2$ and the previously derived facts $I(1)$ and
$T(1,2)$, giving us $S(2) \impliedby True$. The last step again applies
resolution to the instance of the fourth clause with $x := 2$ and the derived
fact $S(2)$ resulting in $False$ clause. 

Therefore the proof gives us the complete trace from initial states resulting with a false clause.


\section{Engine integration}
\noindent The architecture described above allows us to integrate an engine into the Golem solver without modifying it. There are several ways we can do it. 

Writing a library in a different programming language could be an interesting
option, as it would allow us to leverage certain advantages of other
programming languages and make our engine compatible with other solvers and
verification programs. However, integrating C++ with another language may
introduce some performance overhead. This overhead could be caused by complex
data conversions, memory allocation issues, or additional context switching.
The extent of this overhead depends on how we plan to utilize the library. If
the library was a standalone engine that received input from Golem and returned
an SAT result, the overhead would be minimal. However, if we wanted the library to
interact with Golem more, such as by using its SMT solver or storing additional
information about the solving process, the performance could degrade. 

The goal of this work is to create a new efficient engine within the Golem solver and compare its
performance with other Golem engines. As we said above, using or implementing a different SMT solver and isolating the solving process in a library would not
bring any noticeable overhead. Therefore it could look like a feasible
solution. However, choosing this approach would undermine the purpose of this work. We would not be able to compare the performance because the comparison
of different engines would be affected by the performance of their underlying
SMT solvers.

Another option would be to write a C++ library and include it in the project.
The big advantage of this solution would be the option of using the engine library
in other solvers. Using C++ would also eliminate the mentioned downsides of the
first option, but it would still leave some. For example, to keep the engine
universal, we must integrate an SMT solver for satisfiability checking
and interpolation. That, as we have already discussed, would not be an
optimal solution for the purpose of this work. 

The last option would be to utilize Golem for SMT solving. With this approach, we would use C++ to achieve high performance and we would use Golem's integrated SMT solver to eliminate the above-mentioned disadvantages such as not being able to effectively compare engines with different SMT solvers. On the other hand, such an approach would make our engine usable only for the Golem solver and it could not be used as a standalone engine in other solvers.

The most suitable option for achieving our goal is to integrate the engine directly into the Golem solver, following the approach used for its existing engines. This approach allows us to fully leverage OpenSMT, Golem’s integrated SMT solver, for satisfiability checking and interpolation. Additionally, we can take advantage of other Golem’s built-in features, which would simplify development and improve efficiency. Although this method ties the engine to Golem, it is ideal for the work's goal of developing a high-performance engine for Golem and comparing it against its other engines.

