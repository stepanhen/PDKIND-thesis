\chapter{PDKind}

% Describe the individual procedures of the algorithm and in each part analyze how the implementation went. If there were more ways to do it, compare them.
% The last section should describe how we also added validity witnesses because Golem and other engines in it produce them).
\noindent In this chapter, we describe and analyze the PDKind algorithm. In sections 4.1-4.4, we will be using methodology, algorithms, and definitions described in \cite{7886665}.
We will also describe the modifications of the algorithm and the decisions we made when implementing some abstract data structures from the paper.

After reading this chapter, the reader should be able to understand the parts of the PDKind algorithm and orientate in our implementation. 

\section{Induction vs k-Induction}

\noindent This algorithm is a combination of IC3 and k-induction.
IC3 is a commonly used method that uses induction to show a property is invariant by incrementally constructing an inductive strengthening of the property. PDKind breaks IC3 into modules and that allows replacing the induction method with k-induction.


\vspace{\baselineskip}\noindent \textbf{Definition (Induction):} Proves a property \( P \) is invariant by showing:
\begin{itemize}
    \item \textbf{Base Case (init)}: \( P \) holds in the initial state.
    \item \textbf{Inductive Step (cons)}: If \( P \) holds in a state, it holds in the next state.
\end{itemize}


\vspace{\baselineskip}\noindent \textbf{Definition (k-Induction):} Extends traditional induction to consider sequences of \( k \) states. Proves a property \( P \) is invariant by showing:
\begin{itemize}
    \item \textbf{Base Case (k-init)}: \( P \) holds in the first \( k \) states.
    \item \textbf{Inductive Step (k-cons)}: If \( P \) holds in a sequence of \( k \) states, it holds in the next state.
\end{itemize}

This method is more powerful for properties that are not inductive but can be shown to hold over multiple steps.
More precise definitions are shown in \cite{7886665}.

\subsection*{Relative Power}
\noindent With Quantifier Elimination, induction and k-induction have the same deductive power. K-Induction might provide more concise proofs.

Without Quantifier Elimination k-induction can be exponentially more concise than induction. Stronger in certain logical theories like pure Boolean logic or linear arithmetic.

\subsection*{Practical Effectiveness}
\noindent \cite{7886665} shows that k-Induction is effective, especially when combined with algorithms like IC3.
Here we demonstrate its effectiveness on a simple example from the paper\cite{7886665}.

\newpage

\begin{figure}
\begin{mdframed}
\begin{lstlisting}
Invariant Property P: a[0] == 0

Initial States:
i = 0
j = 0
a[0] = 0

Transition:
j = Random() mod (i+1)
i = i + 1 mod N
a[i] = a[j]
\end{lstlisting}
    \caption{Induction vs K-induction example}\label{ex:ind vs kind}
\end{mdframed}
\end{figure}

\vspace{\baselineskip}Consider a transition system depicted in Figure \ref{ex:ind vs kind}.
We can see, that induction cannot prove \( P \) as invariant because \( P \) is not inductive in a single step. P is even k-inductive for $k \leq N$. However, we can use k-induction to prove \( P \) is \( (N+1) \)-inductive, where \( N \) is the length of the array.

\vspace{\baselineskip}\noindent \textbf{Proof:}
\begin{itemize}
    \item Any sequence of $N+1$ states has a transition that resets i to 0.

\item All transitions increment i to at most $N - 1$.

\item Pick j between 0 and i and do a[i] = a[j].

\item If P holds in all states in this sequence and $i = N - 1$ in the last state, then $a[0]=a[1]=\dots=a[N-1]=0$.

\item If $i \neq N - 1$ in the last state, then a[0] is unchanged in next transition.
\end{itemize}
\noindent This concludes that P is $(N + 1)$-inductive.

\hfill\square


\vspace{\baselineskip}This shows that \( k \)-induction can handle properties where simple induction fails, especially in systems with complex state transitions.

\section{Rechability checking procedure}\label{ReachabilityChecking}

\noindent To understand the following, we need to introduce a few definitions.
Let us have a formula in the form 
\begin{equation}
    A(\vec{x}) \wedge T[B]^k(\vec{x},\vec{y},\vec{w}) \wedge C(\vec{y})
\end{equation}
Where for \( k > 1 \), \( T[F]^k(\vec{x}, \vec{x}') \) is defined as
\begin{equation*}
    T(\vec{x}, \vec{w}_1) \land \bigwedge_{i=1}^{k-2} \left( F(\vec{w}_i) \land T(\vec{w}_i, \vec{w}_{i+1}) \right) \land F(\vec{w}_{k-1}) \land T(\vec{w}_{k-1}, \vec{x}')
\end{equation*}
where $\vec{w}$ are state variables in the intermediate states.


\vspace{\baselineskip}\noindent    \textbf{Definition (Interpolant)\cite{7886665}:} If formula (4.1) is unsatisfiable, then $I(\vec{y})$ is an interpolant if
    \begin{enumerate}
       \item \( A(\vec{x}) \land T[B]^k(\vec{x}, \vec{w}, \vec{y}) \Rightarrow I(\vec{y}) \), and
       \item \( I(\vec{y}) \) and \( C(\vec{y}) \) are inconsistent.
\end{enumerate}

\noindent    \textbf{Definition (Generalization)\cite{7886665}:} If formula (4.1) is satisfiable, then $G(\vec{x})$ is a generalization if
    \begin{enumerate}
        \item $G(\vec{x}) \Rightarrow \exists \vec{y}, \vec{w}$  $T[B]^k(\vec{x}, \vec{y}, \vec{w}) \wedge C(\vec{y})$, and
        \item $G(\vec{x})$ and $A(\vec{x})$ are consistent.
\end{enumerate}

\renewcommand{\figurename}{Algorithm}
\begin{figure}[h]
\begin{mdframed}
\begin{algorithmic}[1]
\State \textbf{Input:} Target state $F$, maximum steps $k$
\State \textbf{Data:} Reachability frames $R$, initial states $I$, transition states $T$
\State \textbf{Output:} True if $F$ is reachable in $k$ steps, False otherwise

\If{$k = 0$} 
    \State \Return{CheckSAT($I \wedge T^0 \wedge F$)}
\EndIf

\While{true}
    \If{CheckSAT($R_{k-1} \wedge T \wedge F$)}
        \State $G \gets$ Generalize($R_{k-1}, T, F$)
        \If{Reachable($G, k-1$)}
            \State \Return{true}
        \Else
            \State $E \gets$ Explain($G, k-1$)
            \State $R_{k-1} \gets R_{k-1} \cup E$
        \EndIf
    \Else
        \State \Return{false}
    \EndIf
\EndWhile

\end{algorithmic}
\end{mdframed}
\caption{Reachable method}\label{alg:1}
\end{figure}

\noindent Method shown in Algorithm \ref{alg:1} tries to reach the initial states backwards by using a depth-first search strategy.

To check if \( F \) is reachable from the initial states in \( k \) steps, we first check whether \( F \) is reachable in one transition from the previous frame \( R_{k-1} \). If there is no such transition, then \( F \) is not reachable in \( k \) steps. Otherwise, we get a state that satisfies \( R_{k-1} \) and from which \( F \) is reachable in one step. We then call a generalization procedure, which gives us a formula \( G \), a generalization of the state mentioned above. Then, using a DFS strategy, we recursively check whether \( G \) is reachable from the initial states. If \( G \) is reachable, then \( F \) is also reachable, and the procedure ends. Otherwise, we can learn an explanation and eliminate \( G \) by adding the explanation into the frame \( R_{k-1} \).

In our implementation, we needed to figure out the following: \textbf{Satisfiability checking}, \textbf{Generalization}, \textbf{Explanation} and \textbf{Reachability frames representation}.

\section*{Satisfiability Checking}
\noindent We could use various SMT solvers for satisfiability checking, but Golem already has a wrapper around the OpenSMT solver and uses it in every other engine. Therefore we will use it too for every other satisfiability check, that we will need.

\section*{Generalization}
\noindent Golem doesn't provide us with the generalization method. Still, it has needed components to create our method, as shown in pseudocode Algorithm \ref{alg:2}. It eliminates all variables except the state ones, represented as $\vec{x}$ from the formula \( T[B]^k(\vec{x}, \vec{w}, \vec{y}) \wedge C(\vec{y}) \) to satisfy the generalization definition.
\begin{figure}[H]
\begin{mdframed}
\begin{algorithmic}[1]
\State \textbf{Input:} Model $M$, transition formula $T$, state formula $F$
\State \textbf{Output:} Generalized formula $G$

    \State $StateVars \gets$ GetStateVars()
    \State $G \gets$ KeepOnly(StateVars, $T \wedge F$, $M$)
    \State \Return $G$

\end{algorithmic}
\end{mdframed}
\caption{Generalize method}\label{alg:2}
\end{figure}

\section*{Explanation} \label{Explain}
\noindent Instead of the \texttt{Explain()} method, we will use another feature of the OpenSMT solver and that is interpolation. To get the interpolant, we tell the solver to give us an interpolation of the first two formulas inserted in it, i.e. for \texttt{CheckSat($A \wedge B \wedge C$)} we want interpolation of $A \wedge B$. 


On line 13 in Algorithm \ref{alg:1}, we need an interpolant from the \texttt{CheckSAT()} that happened in the previous \texttt{Reachable()} call. Therefore, we need to modify our \texttt{Reachable()} method on line 17 to also return an interpolant along with the false result. Later, we will see that in our implementation each \texttt{CheckSAT()} has its solver instance. To get the interpolant, we just ask the solver to return it if the \texttt{CheckSAT()} fails.

\section*{Reachability frames representation}\label{RFrames}
\noindent For the \texttt{Reachability frames} representation, we had several choices. The simplest one was to create a list $R$ of formulas (where $R_i := R[i]$) each time we call \texttt{Reachable()} from outside. This approach is simple, yet not efficient since we would be losing the whole \texttt{Reachability frames} we built in each call, instead of reusing it in another call. 

Therefore we will create a better approach and that is a \texttt{Reachability} class, where each instance of this class will hold such list, but calling \texttt{Reachable()} on that instance would only grow the \texttt{Reachability frames} and wouldn't delete it.
\section{Push procedure} \label{Push}

\noindent \textbf{Definition (Induction Frame)\cite{7886665}:}\label{def:IFrame} A set of tuples $F \subset \mathbb{F} \times \mathbb{F}$, where $\mathbb{F}$ is a set of all state formulas in theory $T$, is an induction frame at index $n$ if $(P, \neg P) \in F$ and $\forall (lemma, counterExample) \in F$:
\begin{itemize}
    \item $lemma$ is valid up to $n$ steps and refutes $counterExample$
    \item $counterExample$ states can be extended to a counterexample to $P$.
\end{itemize}

\noindent The procedure shown in Algorithm \ref{alg:3} is the core of the PDKind algorithm, We will break it down into smaller pieces to understand how it works.

The first part starts on line 11 where we need to check if \texttt{lemma} is k-inductive. We can do that by checking if the formula $(F_{ABS} \wedge T_k \wedge \neg lemma)$ is satisfiable. We can notice that the \texttt{CheckSAT} call returns model \texttt{m\_1}. In our implementation, \texttt{CheckSAT} doesn't directly return model, but we can ask the solver to provide one. This is done by initializing a new solver instance for each check and then asking for a model, if the check is true.
If the formula $(F_{ABS} \wedge T_k \wedge \neg lemma)$ isn't satisfiable, we can push a new obligation to our new induction frame \texttt{G} and continue. Else we get the model \texttt{m\_1} and save it for later.

In the second part, on line 16, we check if \texttt{counterExample} is reachable. If it is, we get model \texttt{m\_2} and generalize it to \texttt{g\_2}. We know that from \texttt{g\_2}, we can reach \texttt{\neg P}, so we need to check if \texttt{g\_2} is reachable from initial states. If it is reachable, the property is invalid, and we mark \texttt{isInvalid \gets true}.

On line 19, we can see that \texttt{Reachable()} accepts more arguments and returns more values than shown in Algorithm \ref{alg:1}. This new \texttt{Reachable($i$,$j$,$F$)} method checks if $F$ is reachable in k steps where $i\leq k \leq j$. To achieve this behavior in our implementation, we create a wrapper function that calls \texttt{Reachable($k$, $F$)} in a for loop and returns the first \texttt{k} where the call was successful along with the result. The \texttt{i} return value is an interpolant of the last \texttt{Reachable()} check if the whole check wasn't successful.

If this call wasn't successful, we get an interpolant \texttt{i\_1} and assign it to \texttt{g\_3} which eliminates \texttt{g\_2}. We found a new induction obligation \texttt{(g\_3, g\_2)}, which is a strengthening of \texttt{F}. Now, we can try again with a potential counterexample eliminated.

Last step is to analyze the induction failure. From the first check we have a model \texttt{m\_1}, which is a counterexample to the k-inductiveness of \texttt{lemma}. We again get \texttt{g\_1} as a generalization of \texttt{m\_1} and check if \texttt{g\_1} is reachable from initial states. If it is reachable, we replace \texttt{lemma} with weaker \texttt{\neg counterExample} and push this new obligation to \texttt{F} and \texttt{G}. On the other hand, if \texttt{g\_1} is not reachable, we strengthen \texttt{lemma} with \texttt{g\_3} and push this new obligation to \texttt{F}.
\begin{figure}[H]
\begin{mdframed}
\begin{algorithmic}[1]
\State \textbf{Input:} Induction frame $F$, $n$, $k$
\State \textbf{Output:} Old induction frame $F$, new induction frame $G$, $n_p$, $isInvalid$ 

\State push elements of $F$ to queue $Q$
    \State $G \gets$ \{\}
    \State $n_p \gets n + k$
    \State $invalid \gets false$
    \While {$\neg invalid$ and $Q$ is not empty}
        \State $(lemma, counterExample)  \gets$ $Q$.pop()
        \State $F_{ABS} \gets \bigwedge a_i$ ,where ($a_i$, $b_i$) \in $F$ \forall $i$ \in \{1,...,$F$.length()\}
        \State $T_k \gets T[F_{ABS}]^k$ by definition
        \State $(s_1, m_1) \gets$ CheckSAT($F_{ABS}, T_k, \neg lemma$)
        \If{$\neg s_1$}
            \State $G \gets G \cup (lemma, counterExample)$
            \State Continue
        \EndIf
        \State $(s_2, m_2) \gets$ CheckSAT($F_{ABS} \wedge T_k \wedge counterExample$)
        \If{$s_2$}
            \State $ g_2 \gets $ Generalize($m_2, T_k, counterExample$)
            \State $(r_1, i_1, n_1) \gets $ Reachable($n-k+1$, $n$, $g_2$)
            \If{$r_1$}
                \State $ isInvalid \gets true $
                \State Continue
            \Else
                \State $g_3 \gets i_1$
                \State $F \gets F \cup (g_3, g_2)$
                \State $Q$.push($(g_3, g_2)$)
                \State $Q$.push($lemma, counterExample$)
                \State Continue
            \EndIf
        \EndIf
        \State $g_1 \gets$ Generalize($m_1$, $T_k$, $\neg lemma$)
        \State $(r_2, i_2, n_2) \gets $ Reachable($n-k+1$, $n$, $g_1$)
        \If{$r_2$}
            \State $(r_3, i_3, n_3) \gets $ Reachable($n+1$, $n_2 + k$, $g_1$)
            \State $n_p \gets $ Min($n_p$, $n_3$)
            \State $F \gets$ $F$ $\cup $ ($\neg counterExample$, $counterExample$)
            \State $G \gets$ $G$ $\cup $ ($\neg counterExample$, $counterExample$)
        \Else
            \State $g_3 \gets$ $i_2 \wedge lemma$
            \State $F \gets$ $F$ $\cup $ ($g_3$, $counterExample$)
            \State $F \gets$ $F$ $\setminus $ ($lemma$, $counterExample$)
            \State $Q$.push(($g_3$, $counterExample$))
        \EndIf
        \State \Return ($F$, $G$, $n_p$, $isInvalid$)
    \EndWhile
        
\end{algorithmic}
\end{mdframed}
\caption{Push procedure}\label{alg:3}
\end{figure}


\newpage 
\section{PD-Kind procedure}

\noindent The main PDKind procedure shown in Algorithm \ref{alg:4} checks if propert \texttt{P} is invariant by iteratively calling the \texttt{Push} procedure to find a k-inductive strengthening of \texttt{P} for some $1 \leq k \leq n + 1$. The strengthening \texttt{G} is k-inductive and if \texttt{F == G}, then \texttt{P} is invariant and we return \texttt{SAFE}. If the \texttt{Push} procedure marks \texttt{isInvalid} as \texttt{True}, the property is not invariant and we return \texttt{UNSAFE}. Otherwise, we update \texttt{n} and repeat the loop.

In our implementation, the property \texttt{P} is a negation of a query, that we get on the input, which represents bad states. We also need to check if the initial states are empty, which would result in \texttt{SAFE}, or if the query holds in the initial states, which would result in \texttt{UNSAFE}.

\begin{figure}[h]
\begin{mdframed}
\begin{algorithmic}[1]
\State \textbf{Input:} Initial states $I$, transition formula $T$, property $P$
\State \textbf{Output:} Retrun UNSAFE if P is invalid or SAFE when there is no inductive strengthening left

\State $n \gets 0$
    \State $F \gets (P, \neg P)$
    \While{true}
        \State $k \gets n + 1$
        \State $(F, G, n_p, isInvalid) \gets$ Push($F$, $n$, $k$)
        \If{$isInvalid$}
        \State \Return UNSAFE
        \EndIf
        \If{$F = G$}
        \State \Return SAFE
        \EndIf
        \State $n \gets n_p$
        \State $F \gets G$
    \EndWhile

\end{algorithmic}
\end{mdframed}
\caption{Main PD-Kind procedure}\label{alg:4}
\end{figure}

\section{Validity checking}
\noindent In many cases, it is often required to provide a witness to the answer obtained from solving the CHC satisfiability problem. In software verification, a satisfiability witness corresponds to a program invariant, and an unsatisfiability witness corresponds to counterexample paths. Generally, a satisfiability witness is a model that provides an interpretation of all CHC predicates and variables that satisfy all the clauses. An unsatisfiability witness is a proof presented as a sequence of derivations of ground instances of the predicates, where for the proof to be valid, each premise must be a conclusion of some previously derived step.

In Golem \cite{blicha_golem_2023}, each engine provides a validity witness when the option \texttt{--print-witness} is used. To follow the structure Golem has, we need to implement such an option for our PDKind engine as well.

\section*{UNSAT witness} \label{UNSATWit}
\noindent First, we will describe the implementation of the unsatisfiability witness in our engine. The goal is to generate paths to counterexamples during the CHC satisfiability solving process.

To do that, we utilize a function in the Golem solver, which can generate the path to a counterexample based on the number of steps required to reach the counterexample, for clarity, we will call them \texttt{steps to the counterexample}. This allows us to only keep track of the \texttt{steps to a counterexample} for each potential counterexample we encounter during the CHC solving. The \texttt{steps to a counterexample} of a potential counterexample is a number of steps needed to reach the counterexample from the potential counterexample. To do this, we take the \texttt{Induction frame} \texttt{(lemma, counterExample)} and create a structure for the \texttt{counterExample}, which will hold the formula and the \texttt{steps to a counterxample}.

In the next step, we need to correctly assign the \texttt{steps to a counterexample} to each counterexample, that we find. In Algorithm \ref{alg:3}, we note that a new counterexample \texttt{g\_2} is created only on line 18, which we then use in the else branch starting on line 23. The \texttt{steps to a counterexample} assigned to \texttt{g\_2} is the \texttt{steps to a counterexample} of \texttt{counterExample} increased by \texttt{k}.

Finally, we need to modify \texttt{Push()} to return the \texttt{steps to a counterexample}. On line 21, when we encounter a reachable counterexample \texttt{g\_2}, we assign the \texttt{steps to a counterexample} returned by \texttt{Push()} to be the \texttt{steps to a counterexample} of \texttt{g\_2}.

\section*{SAT witness}
\noindent In this section, we will describe the implementation of the satisfiability witness in the PDKind engine. The aim is to construct an inductive invariant during the CHC satisfiability solving process.

In the \texttt{Push()} procedure described in Algorithm \ref{alg:3}, we are constructing an \texttt{Induction Frame}, which is a set of tuples \texttt{(lemma, counterExample)}, where the \texttt{lemma} holds for \texttt{n} steps and refutes the \texttt{counterExample}. After the solving procedure is finished, we end up on line 12 of the PDKind procedure in Algorithm \ref{alg:4} because we are constructing the satisfiability witness. We can then take the final \texttt{Induction Frame} and form a conjunction of all the lemmas within it. This gives us an n-inductive invariant, as the lemmas hold for \texttt{n} steps, and there is no other strengthening of the \texttt{Induction Frame}, as indicated by \( F = G \) at line 11.

The final step is to transform the n-inductive invariant into an inductive invariant. To do that, we utilize the Golem solver's function \texttt{kinductiveToInductive()}, which takes the n-inductive invariant, \texttt{n}, and the system and returns an inductive invariant. This invariant is the validity witness for the \texttt{SAT} answer.

