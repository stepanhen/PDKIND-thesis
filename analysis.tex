\chapter{PDKind}
\label{PD-Kind}
% Describe the individual procedures of the algorithm and in each part analyze how the implementation went. If there were more ways to do it, compare them.
% The last section should describe how we also added validity witnesses because Golem and other engines in it produce them).
\noindent PDKind is an IC3\cite{6148908}-based algorithm, that separates reachability checking and induction reasoning, allowing the induction core to be replaced with k-induction.

In this chapter, we describe and analyze the PDKind algorithm. In sections 4.1-4.3, we will be using methodology, algorithms, and definitions described in \cite{7886665}.
After reading this chapter, the reader should be able to understand the parts of the PDKind algorithm and orientate in our implementation.
%We will also describe the modifications of the algorithm and the decisions we made when implementing some abstract data structures from the paper.


\section{PDKind Main Procedure}  \label{sec:main} The PDKind algorithm aims
to prove a safety property $P$ is invariant or to find a counterexample. It
achieves this by separating the process into two subprocedures, which work together within the main
loop. Before diving into the algorithm, we introduce the key data structures
and subprocedures it uses. \textit{Reachability frames} for the reachability
checking, and \textit{Induction frames} for managing induction obligations.

\textbf{Reachability frames} $R_0, R_1, \dots, R_k$ are used to guide the
reachability checking \ref{ReachabilityChecking}. Each frame $R_i$ is an
over-approximation of the set of states from which the target formula is
reachable in exactly $k - i$ steps. In particular, $R_k$ corresponds to the set
of states satisfying the
target formula (the “bad” states), and $R_0$ represents states that can reach
the target in $k$ steps. These reachability frames help avoid redundant work by
caching which portions of the state space have already been shown to be
unreachable, thereby preventing the algorithm from re-exploring those paths.

\textbf{Induction frames}, on the other hand, are used by the 
Push subprocedure in section \ref{Push} to store
and refine candidate invariants. An induction frame $F$ at a given depth
$n$ is essentially a set of pairs $(\mathit{lemma}, \mathit{counterExample})$,
where each $\mathit{lemma}$ is a formula (state property) that is supposed to
hold as an invariant up to depth $n$, and $\mathit{counterExample}$ is a
description of states that $\mathit{lemma}$ aims to exclude at depth $n$.
Initially, the induction frame is initialized with the pair $(P, \neg P)$, meaning
our starting candidate invariant is $P$ itself, and the “counterexample” to
this invariant is $\neg P$ (the property’s negation). This indicates that until
we strengthen $P$, the primary bad state to consider is any state satisfying
$\neg P$.

Using these structures, the main PDKind procedure coordinates the
model-checking process. In each iteration, it calls the reachability checking and push
subprocedures to either extend the proof of $P$ or find a real counterexample.
Algorithm \ref{alg:4} outlines this top-level procedure. The main loop works as
follows: we maintain an induction frame $F$ (starting with $(P, \neg P)$) at
depth $n$ (starting at $n=0$). At each step, we pick a $k$ (typically $k =
n+1$) and invoke the Push procedure on the current frame $F$ to attempt to find
a $k$-inductive strengthening of $P$. In other words, Push will try to
either prove that $F$ (or an extension of $F$) is inductive for $k$ steps, or
identify how $F$ fails and strengthen it by adding new lemmas.

The Push procedure \ref{Push} internally uses the reachability
checking procedure to check whether certain potential counterexample states are
actually reachable from the initial state. If the Push procedure discovers a
concrete counterexample trace from an initial state to a bad state, it will
mark the property as invalid. Otherwise, Push returns a strengthened set of
lemmas (a new induction frame $G$) that is valid up to a higher depth.

After each call to Push, the main algorithm examines the results:
\begin{itemize} 
    \item If Push reported that the property is invalid (i.e., a
counterexample was found), then the main procedure terminates and returns
\textsf{UNSAFE}, as we have proof that $P$ is not an invariant. 
    \item If the
new frame $G$ returned by Push is identical to the old frame $F$ (i.e.\ $G =
F$, meaning Push did not add any new lemmas), then $F$ is already
$k$-inductive. In this case, no further strengthening is required—$P$ has been
proven invariant—so the algorithm returns \textsf{SAFE}. 
    \item Otherwise (the
property is not yet proven, but no counterexample was found), the algorithm
increases the induction depth and continues. We update $n$ to the new value
$n_p$ provided by Push (which is at least $n+1$), set $F := G$ (using the
strengthened frame as the new current frame), and repeat the loop for the next
iteration. 
\end{itemize}

In this way, the PDKind main procedure iteratively strengthens the candidate
invariant $P$ with new lemmas and deepens the search, until either a valid
inductive invariant is achieved or a real counterexample is discovered.

\begin{figure}[H]
\begin{mdframed}
\begin{algorithmic}[1]
\State \textbf{Input:} Initial states $I$, transition formula $T$, property $P$
\State \textbf{Output:} Return UNSAFE if P is invalid or SAFE when there is no inductive strengthening left

\State $n \gets 0$
    \State $F \gets (P, \neg P)$
    \While{true}
        \State $k \gets n + 1$
        \State $(F, G, n_p, isInvalid) \gets$ Push($F$, $n$, $k$)
        \If{$isInvalid$}
        \State \Return UNSAFE
        \EndIf
        \If{$F = G$}
        \State \Return SAFE
        \EndIf
        \State $n \gets n_p$
        \State $F \gets G$
    \EndWhile

\end{algorithmic}
\end{mdframed}
\caption{Main PDKind procedure}\label{alg:4}
\end{figure}


\section{Reachability checking procedure}\label{ReachabilityChecking}

\noindent To understand the following, we need to introduce a few definitions.

\vspace{\baselineskip}\begin{definition}Given a transition $T$ and a state formula $F$.
For \( k > 1 \), \( T[F]^k(\vec{x}, \vec{x}') \) is defined as
                          \begin{equation*}
                              T(\vec{x}, \vec{w}_1) \land \bigwedge_{i=1}^{k-2} \left( F(\vec{w}_i) \land T(\vec{w}_i, \vec{w}_{i+1}) \right) \land F(\vec{w}_{k-1}) \land T(\vec{w}_{k-1}, \vec{x}')
                          \end{equation*}
                          This represents the unrolling of the transition relation $T$ over $k$ steps, where $F$ holds in the intermediate states $\vec{w}$.
                          \label{Transition}
\end{definition}

\vspace{\baselineskip}\noindent
Not only in this section, but in the whole algorithm we will be using two important techniques, interpolation and generalization. 
Let us have a formula in the form
\begin{equation}
    A(\vec{x}) \wedge T[B]^k(\vec{x},\vec{w},\vec{y}) \wedge C(\vec{y})
\end{equation}
where \( T[B]^k(\vec{x}, \vec{w}, \vec{y}) \) follows the previous definition \ref{Transition}, and \( \vec{x} \), \( \vec{w} \), and \( \vec{y} \) represent the initial, intermediate, and final state variables, respectively.


\vspace{\baselineskip}
\begin{definition}{\textbf{(Interpolant)}}
    \cite{7886665} If formula (4.1) is unsatisfiable, then $I(\vec{y})$ is an interpolant if
    \label{interpolation}
    \begin{enumerate}
        \item \( A(\vec{x}) \land T[B]^k(\vec{x}, \vec{w}, \vec{y}) \Rightarrow I(\vec{y}) \), and
        \item \( I(\vec{y}) \) and \( C(\vec{y}) \) are inconsistent.
    \end{enumerate}
\end{definition}
Interpolants approximate the set of reachable states from a source \( A(\vec{x}) \) and exclude a target \( C(\vec{y}) \). They are used when the formula is unsatisfiable and help rule out unreachable parts of the state space by over-approximating the reachable states.


\vspace{\baselineskip}
\begin{definition}{\textbf{(Generalization)}}
    \cite{7886665} If formula (4.1) is satisfiable, then $G(\vec{x})$ is a generalization if
    \label{generalization}
    \begin{enumerate}
        \item $G(\vec{x}) \Rightarrow \exists \vec{y}, \vec{w}$  $T[B]^k(\vec{x}, \vec{y}, \vec{w}) \wedge C(\vec{y})$, and
        \item $G(\vec{x})$ and $A(\vec{x})$ are consistent.
    \end{enumerate}
\end{definition}
When the formula is satisfiable, we generalize the final state to an under-approximation that still satisfies the formula but covers more states. This helps us work with a broader set of counterexamples and makes the recursive check more efficient.


\renewcommand{\figurename}{Algorithm}
\begin{figure}[H]
    \begin{mdframed}
        \begin{algorithmic}[1]
            \State \textbf{Input:} Target state $F$, maximum steps $k$
            \State \textbf{Data:} Reachability frames $R$, initial states $I$, transition states $T$
            \State \textbf{Output:} True if $F$ is reachable in $k$ steps, False otherwise

            \If{$k = 0$}
                \State \Return{CheckSAT($I \wedge T^0 \wedge F$)}
            \EndIf

            \While{true}
                \If{CheckSAT($R_{k-1} \wedge T \wedge F$)}
                    \State $G \gets$ Generalize($R_{k-1}, T, F$)
                    \If{Reachable($G, k-1$)}
                        \State \Return{true}
                    \Else
                        \State $E \gets$ Explain($G, k-1$)
                        \State $R_{k-1} \gets R_{k-1} \cup E$
                    \EndIf
                \Else
                    \State \Return{false}
                \EndIf
            \EndWhile

        \end{algorithmic}
    \end{mdframed}
    \caption{Reachable method}
    \label{alg:1}
\end{figure}

\vspace{\baselineskip}
\noindent The core part of PDKind is a depth-first reachability procedure that checks whether the target formula is reachable in exactly \( k \) steps. Instead of searching forward from the initial states, it works backwards from the target and tries to find a trace of the required length that can be extended to the initial states.

To make the search more efficient, the algorithm maintains reachability frames \( R_0, \dots, R_k \), where each frame \( R_i \) over-approximates the set of states that can reach the target in exactly \( k - i \) steps. These frames help avoid repeating work and rule out parts of the state space that have already been shown to be unreachable.

When a satisfying trace is found, the final state is generalized into an under-approximation that still satisfies the formula but includes more states. The recursive check is then applied to this generalization. If the trace can’t be extended to the initial states, the algorithm learns an interpolant that blocks it and adds it to the corresponding frame to avoid exploring similar paths again.

This procedure is used repeatedly in PDKind and forms the main loop of the algorithm. The pseudocode \ref{alg:1} describes it in more detail.



\section{Push procedure} \label{Push}
We have already metioned that the Push procedure utilizes data structure called Induction frame. Here we formally define it. 

\vspace{\baselineskip}
\begin{definition}{\textbf{(Induction Frame)}}
    \cite{7886665}\label{def:IFrame} A set of tuples $F \subset \mathbb{F} \times \mathbb{F}$, where $\mathbb{F}$ is a set of all state formulas in theory $T$, is an induction frame at index $n$ if $(P, \neg P) \in F$ and $\forall (lemma, counterExample) \in F$:
    \begin{itemize}
        \item $lemma$ is valid up to $n$ steps  and blocks the states described by \\ \( counterExample \), preventing them from occurring at depth \( n \)
        \item $counterExample$ states can be extended to a counterexample to $P$.
    \end{itemize}
\end{definition}
In essence, an induction frame tracks the progressive strengthening of an
invariant over multiple iterations. Each tuple in $F$ consists of a candidate
invariant formula (the lemma) and a specific “bad” state (the counterexample)
that the lemma is meant to eliminate from consideration at the current depth.
As the algorithm runs, these frames are updated: formulas are refined to rule
out newly discovered bad states, strengthening the invariant.

\vspace{\baselineskip}
\noindent Induction frame tracks the strengthening of invariants over number of iterations. Every frame consists of a formula that has been refined to eliminate an invalid state.

The Push procedure \ref{alg:3} is the core of the PDKind algorithm’s
inductive reasoning stage. It takes as input the current induction frame $F$ at
index $n$ (which contains the current set of lemmas and their associated
counterexamples) and attempts to push these lemmas to a higher inductive
step (specifically, up to $k = n+1$) by analyzing counterexamples. In other
words, Push tries to either prove each lemma holds one step further or to find
why it fails and strengthen the frame accordingly. Throughout this process,
Push alternates between generalization and reachability checks.
We now break down the Push algorithm step by step.

\textbf{(1) Initial $k$-inductiveness check:} \\
For each obligation $(\mathit{lemma}, \mathit{counterExample})$ in the frame $F$, Push first checks
whether $\mathit{lemma}$ is $k$-inductive (see Definition \ref{Def:kind}). This is done by
querying the SMT solver to check the satisfiability of
$F_{ABS} \wedge T^k \wedge ¬lemma$, where $F_{ABS}$ denotes the
conjunction of all lemmas in $F$ 
and $T^k$ represents $k$ steps of the transition relation. If this formula is
unsatisfiable, it means that $\mathit{lemma}$ is already strong enough up to $n+k$ steps, and we can simply
carry this obligation into the new induction frame $G$ and continue to the next
obligation. If instead the formula is satisfiable, the solver produces a model
$m_1$ witnessing the failure of $k$-inductiveness (meaning $m_1$ is a state at
depth $k$ where $\mathit{lemma}$ does not hold under the current assumptions).
We save this counterexample state $m_1$ for later analysis and do not yet add
$\mathit{lemma}$ to $G$.

\textbf{(2) Counterexample reachability check:} \\
Next, the Push procedure examines the current $\mathit{counterExample}$ associated with the lemma. It
checks whether this $\mathit{counterExample}$ could actually occur in a
concrete execution. This is done by checking satisfiability of
$F_{ABS} \wedge T^k \wedge counterExample$, which asks whether there
exists a trace of length $k$ consistent with our current abstraction
$F_{ABS}$ that ends in a state satisfying $\mathit{counterExample}$. If
this check returns satisfiable, we obtain a witness model $m_2$ that
demonstrates a state at depth $k$ matching the counterexample. We then
generalize $m_2$ to a formula $g_2$. At this point, we know that from any
state described by $g_2$, we can reach a violation of $P$ (because $g_2$
over-approximates the counterexample at depth $k$, which leads to $\neg P$ at depth $k$).
Now we must check if any initial state can lead to a state in $g_2$ within $k$
steps. We call $\textsc{Reachable}(i, j, g_2)$ with to see if $g_2$ lies on some path from the initial
states of length between $i$ and $j$ (essentially up to $n$ steps). If this
reachability check returns true, then we have discovered a concrete
counterexample execution from an initial state to a state that leads to $\neg
P$. In this case, the property $P$ is truly violated; we set
$\mathit{isInvalid} := \text{true}$, indicating that a counterexample has been
found, and we can break out of the Push loop early. If, on the other hand, the
reachability procedure determines that $g_2$ is not reachable from any
initial state, then $m_2$ was a spurious counterexample under the current abstraction. 
The reachability check will also provide an interpolant  $i_1$ in this
scenario, which is a formula that separates the initial states from the
generalized counterexample $g_2$. Specifically, $i_1$ blocks all
paths to $g_2$. We use this $i_1$ to refine our invariant by
strengthening the current lemma. We then add the new obligation
$(i_1, g_2)$ to the current frame $F$. By doing this, we have eliminated the potential
counterexample $g_2$, and we “try again” with a strengthened invariant.

\textbf{(3) Induction failure analysis:} \\
After handling the $\mathit{counterExample}$, the Push procedure returns to address the initial
inductiveness failure captured by $m_1$ from step (1). Now we consider
the generalized form of $m_1$, which is $g_1 = \textsc{Generalize}(m_1, T^k,
\neg \mathit{lemma})$, which represents a set of states at depth $k$ where
$\mathit{lemma}$ could fail. We need to check whether this scenario is genuine
or not. Using the reachability check again, we ask if $g_1$ is reachable from
the initial state. If $g_1$ is reachable from some initial state, then the failure of $\mathit{lemma}$ at
depth $k$ corresponds to a real counterexample scenario. We take the original
$\mathit{counterExample}$ and add a new obligation $(\neg \mathit{counterExample}, \mathit{counterExample})$ to both
the current frame $F$ and the new frame $G$. This new lemma $\neg
\mathit{counterExample}$ is a weaker condition which is trivially true up to depth $n$ but will
block the counterexample at depth $n+1$. By pushing $(\neg
\mathit{counterExample}, \mathit{counterExample})$ into $F$ and $G$, we ensure
that this particular bad state is explicitly ruled out in the future, and we
remove the original stronger (but non-inductive) lemma from consideration.

If $g_1$ is not reachable from any initial state, we can
salvage the lemma by strengthening it. We compute $g_3$ (an
interpolant-based strengthening) by combining $i_2$ with the
original $\mathit{lemma}$ to form a new lema that avoids the counterexample. We replace the old
$(\mathit{lemma}, \mathit{counterExample})$ in $F$ with $(g_3 \land
\mathit{lemma}, \mathit{counterExample})$, and push this new tuple back into
$F$ for further processing. This way, the lemma has been fixed to be true up
to depth $n+1$.

After these steps, the Push procedure continues to process any remaining
obligations in $F$ (each of which goes through the same sequence of checks and
potential updates). If at any point a real counterexample was found
($\mathit{isInvalid} = \text{true}$), Push will terminate early and propagate
that information to the main procedure. Otherwise, once all obligations are
handled, Push produces a new induction frame $G$ containing all the lemmas that
have been confirmed or strengthened for the next depth. It also provides an
updated index $n_p$ the new depth up to which $G$’s lemmas are valid. The
original frame $F$ may also have been updated with strengthened lemmas during
this process. Finally, Push returns $(F, G, n_p, \mathit{isInvalid})$ to the
main PDKind procedure. The main loop \ref{sec:main} will then use these
results to decide whether to declare the property proven (if $F = G$), to
declare it refuted (if $\mathit{isInvalid}$ is true), or to continue the
process with $F := G$ and $n := n_p$ for another iteration.

\begin{figure}[H]
    \begin{mdframed}
        \begin{algorithmic}[1]
            \State \textbf{Input:} Induction frame $F$, $n$, $k$
            \State \textbf{Output:} Old induction frame $F$, new induction frame $G$, $n_p$, $isInvalid$

            \State push elements of $F$ to queue $Q$
            \State $G \gets$ \{\}
            \State $n_p \gets n + k$
            \State $invalid \gets false$
            \While {$\neg invalid$ and $Q$ is not empty}
                \State $(lemma, counterExample)  \gets$ $Q$.pop()
                \State $F_{ABS} \gets \bigwedge a_i$ ,where ($a_i$, $b_i$) \in $F$ \forall $i$ \in \{1,...,$F$.length()\}
                \State $T_k \gets T[F_{ABS}]^k$ by definition
                \State $(s_1, m_1) \gets$ CheckSAT($F_{ABS}, T_k, \neg lemma$)
                \If{$\neg s_1$}
                    \State $G \gets G \cup (lemma, counterExample)$
                    \State Continue
                \EndIf
                \State $(s_2, m_2) \gets$ CheckSAT($F_{ABS} \wedge T_k \wedge counterExample$)
                \If{$s_2$}
                    \State $ g_2 \gets $ Generalize($m_2, T_k, counterExample$)
                    \State $(r_1, i_1, n_1) \gets $ Reachable($n-k+1$, $n$, $g_2$)
                    \If{$r_1$}
                        \State $ isInvalid \gets true $
                        \State Continue
                    \Else
                        \State $F \gets F \cup (i_1, g_2)$
                        \State $Q$.push($(i_1, g_2)$)
                        \State $Q$.push($lemma, counterExample$)
                        \State Continue
                    \EndIf
                \EndIf
                \State $g_1 \gets$ Generalize($m_1$, $T_k$, $\neg lemma$)
                \State $(r_2, i_2, n_2) \gets $ Reachable($n-k+1$, $n$, $g_1$)
                \If{$r_2$}
                    \State $(r_3, n_3) \gets $ Reachable($n+1$, $n_2 + k$, $g_1$)
                    \State $n_p \gets $ Min($n_p$, $n_3$)
                    \State $F \gets$ $F$ $\cup $ ($\neg counterExample$, $counterExample$)
                    \State $G \gets$ $G$ $\cup $ ($\neg counterExample$, $counterExample$)
                \Else
                    \State $g_3 \gets$ $i_2 \wedge lemma$
                    \State $F \gets$ $F$ $\cup $ ($g_3$, $counterExample$)
                    \State $F \gets$ $F$ $\setminus $ ($lemma$, $counterExample$)
                    \State $Q$.push(($g_3$, $counterExample$))
                \EndIf
                \State \Return ($F$, $G$, $n_p$, $isInvalid$)
            \EndWhile

        \end{algorithmic}
    \end{mdframed}
    \caption{Push procedure}\label{alg:3}
\end{figure}


