\chapter{PDKind}

% Describe the individual procedures of the algorithm and in each part analyze how the implementation went. If there were more ways to do it, compare them.
% The last section should describe how we also added validity witnesses because Golem and other engines in it produce them).
\noindent PDKind is an IC3\cite{6148908}-based algorithm, that separates reachability checking and induction reasoning, allowing the induction core to be replaced with k-induction.

In this chapter, we describe and analyze the PDKind algorithm. In sections 4.1-4.3, we will be using methodology, algorithms, and definitions described in \cite{7886665}.
After reading this chapter, the reader should be able to understand the parts of the PDKind algorithm and orientate in our implementation.
%We will also describe the modifications of the algorithm and the decisions we made when implementing some abstract data structures from the paper.




\section{Reachability checking procedure}\label{ReachabilityChecking}

\noindent To understand the following, we need to introduce a few definitions.

\vspace{\baselineskip}\noindent Let us have a formula in the form

\begin{equation}
    A(\vec{x}) \wedge T[B]^k(\vec{x},\vec{y},\vec{w}) \wedge C(\vec{y})
\end{equation}

\vspace{\baselineskip}\begin{definition}
                          For \( k > 1 \), \( T[F]^k(\vec{x}, \vec{x}') \) is defined as
                          \begin{equation*}
                              T(\vec{x}, \vec{w}_1) \land \bigwedge_{i=1}^{k-2} \left( F(\vec{w}_i) \land T(\vec{w}_i, \vec{w}_{i+1}) \right) \land F(\vec{w}_{k-1}) \land T(\vec{w}_{k-1}, \vec{x}')
                          \end{equation*}

                          \noindent where $\vec{w}$ are state variables in the intermediate states.
\end{definition}


\vspace{\baselineskip}
\begin{definition}{\textbf{(Interpolant)}}
    \cite{7886665} If formula (4.1) is unsatisfiable, then $I(\vec{y})$ is an interpolant if
    \label{interpolation}
    \begin{enumerate}
        \item \( A(\vec{x}) \land T[B]^k(\vec{x}, \vec{w}, \vec{y}) \Rightarrow I(\vec{y}) \), and
        \item \( I(\vec{y}) \) and \( C(\vec{y}) \) are inconsistent.
    \end{enumerate}
\end{definition}

\vspace{\baselineskip}
\begin{definition}{\textbf{(Generalization)}}
    \cite{7886665} If formula (4.1) is satisfiable, then $G(\vec{x})$ is a generalization if
    \label{generalization}
    \begin{enumerate}
        \item $G(\vec{x}) \Rightarrow \exists \vec{y}, \vec{w}$  $T[B]^k(\vec{x}, \vec{y}, \vec{w}) \wedge C(\vec{y})$, and
        \item $G(\vec{x})$ and $A(\vec{x})$ are consistent.
    \end{enumerate}
\end{definition}

\renewcommand{\figurename}{Algorithm}
\begin{figure}[h]
    \begin{mdframed}
        \begin{algorithmic}[1]
            \State \textbf{Input:} Target state $F$, maximum steps $k$
            \State \textbf{Data:} Reachability frames $R$, initial states $I$, transition states $T$
            \State \textbf{Output:} True if $F$ is reachable in $k$ steps, False otherwise

            \If{$k = 0$}
                \State \Return{CheckSAT($I \wedge T^0 \wedge F$)}
            \EndIf

            \While{true}
                \If{CheckSAT($R_{k-1} \wedge T \wedge F$)}
                    \State $G \gets$ Generalize($R_{k-1}, T, F$)
                    \If{Reachable($G, k-1$)}
                        \State \Return{true}
                    \Else
                        \State $E \gets$ Explain($G, k-1$)
                        \State $R_{k-1} \gets R_{k-1} \cup E$
                    \EndIf
                \Else
                    \State \Return{false}
                \EndIf
            \EndWhile

        \end{algorithmic}
    \end{mdframed}
    \caption{Reachable method}\label{alg:1}
\end{figure}

\vspace{\baselineskip}
\noindent Method shown in Algorithm \ref{alg:1} tries to reach the initial states backwards by using a depth-first search strategy.

To check if \( F \) is reachable from the initial states in \( k \) steps, we
first check whether \( F \) is reachable in one transition from the previous
frame \( R_{k-1} \). If there is no such transition, then \( F \) is not
reachable in \( k \) steps. Otherwise, we get a state that satisfies \( R_{k-1}
\) and from which \( F \) is reachable in a single step. 

We then call a
generalization procedure, which produces a formula \( G \) (see Definition
\ref{generalization}). Here, generalization helps avoid redundant checks by creating an abstraction of the counterexamples. Instead of storing the given state, it generates a formula covering multiple possible counterexamples, which reduces the number of times the reachability procedure gets called.

Then, using a DFS strategy, we recursively
check whether \( G \) is reachable from the initial states.
If \( G \) is reachable, then \( F \) is also reachable, and the procedure terminates. Otherwise,
we can learn an explanation through interpolation (see Definition
\ref{interpolation}) and eliminate \( G \) by adding the
explanation into the frame \( R_{k-1} \).


\section{Push procedure} \label{Push}

\begin{definition}{\textbf{(Induction Frame)}}
    \cite{7886665}\label{def:IFrame} A set of tuples $F \subset \mathbb{F} \times \mathbb{F}$, where $\mathbb{F}$ is a set of all state formulas in theory $T$, is an induction frame at index $n$ if $(P, \neg P) \in F$ and $\forall (lemma, counterExample) \in F$:
    \begin{itemize}
        \item $lemma$ is valid up to $n$ steps and refutes $counterExample$
        \item $counterExample$ states can be extended to a counterexample to $P$.
    \end{itemize}
\end{definition}

\vspace{\baselineskip}
\noindent Induction frame tracks the strengthening of invariants over number of iterations. Every frame consists of a formula that has been refined to eliminate an invalid state.

The procedure shown in Algorithm \ref{alg:3} is the core of the PDKind algorithm.
It takes the invariant and tries to construct a stronger inductive proof by analyzing its counterexamples and strengthening constraints, while switching between \\
generalization and reachability checking.
We will break it down into smaller pieces to understand how it works.


The first step (line 11) is to check whether \( lemma \) is
\( k \)-inductive (see Definition \ref{Def:kind}). This is done by checking if the formula $(F_{ABS} \wedge T_k
\wedge \neg lemma)$ is satisfiable. We can notice, that \( CheckSAT \) call returns model $m_1$ which
is an assignment of values to variables that satisfies the formula.
If the formula $(F_{ABS} \wedge T_k \wedge \neg lemma)$ isn't satisfiable, we
can push a new obligation to our new induction frame \( G \) and continue.
Otherwise, we get the model $m_1$ and save it for later.

In the second part (line 16), we check if the \( counterExample \) is
reachable. If it is, we get model $m_2$ and generalize it to
$g_2$. We know, that we can reach \( \neg P \) from $g_2$,
so we need to check if $g_2$ is reachable from the initial states. If it is
reachable, the property is invalid, and we mark \( isInvalid \gets true \).


On line 19, we can see that \( Reachable() \) accepts more arguments and
returns more values than shown in Algorithm
\ref{alg:1}. This new
$Reachable(i,j,F)$ method checks if $F$ is reachable in \( k \) steps
where $i\leq k \leq j$.
If this call wasn't successful, we get an interpolant $i_1$ and assign
it to $g_3$ which eliminates $g_2$. We found a new induction
obligation $(g_3, g_2)$, which is a strengthening of \( F \). Now,
we can try again with a potential counterexample eliminated.

Last step is to analyze the induction failure. From the first check we have a
model $m_1$, which is a counterexample to the k-inductiveness of
\( lemma \). We again get $g_1$ as a generalization of $m_1$
and check if $g_1$ is reachable from initial states. If it is
reachable, we replace \( lemma \) with weaker \( \neg counterExample \)
and push this new obligation to \( F \) and \( G \). On the other hand,
if $g_1$ is not reachable, we strengthen \( lemma \) with
$g_3$ and push this new obligation to \( F \).

\begin{figure}[H]
    \begin{mdframed}
        \begin{algorithmic}[1]
            \State \textbf{Input:} Induction frame $F$, $n$, $k$
            \State \textbf{Output:} Old induction frame $F$, new induction frame $G$, $n_p$, $isInvalid$

            \State push elements of $F$ to queue $Q$
            \State $G \gets$ \{\}
            \State $n_p \gets n + k$
            \State $invalid \gets false$
            \While {$\neg invalid$ and $Q$ is not empty}
                \State $(lemma, counterExample)  \gets$ $Q$.pop()
                \State $F_{ABS} \gets \bigwedge a_i$ ,where ($a_i$, $b_i$) \in $F$ \forall $i$ \in \{1,...,$F$.length()\}
                \State $T_k \gets T[F_{ABS}]^k$ by definition
                \State $(s_1, m_1) \gets$ CheckSAT($F_{ABS}, T_k, \neg lemma$)
                \If{$\neg s_1$}
                    \State $G \gets G \cup (lemma, counterExample)$
                    \State Continue
                \EndIf
                \State $(s_2, m_2) \gets$ CheckSAT($F_{ABS} \wedge T_k \wedge counterExample$)
                \If{$s_2$}
                    \State $ g_2 \gets $ Generalize($m_2, T_k, counterExample$)
                    \State $(r_1, i_1, n_1) \gets $ Reachable($n-k+1$, $n$, $g_2$)
                    \If{$r_1$}
                        \State $ isInvalid \gets true $
                        \State Continue
                    \Else
                        \State $g_3 \gets i_1$
                        \State $F \gets F \cup (g_3, g_2)$
                        \State $Q$.push($(g_3, g_2)$)
                        \State $Q$.push($lemma, counterExample$)
                        \State Continue
                    \EndIf
                \EndIf
                \State $g_1 \gets$ Generalize($m_1$, $T_k$, $\neg lemma$)
                \State $(r_2, i_2, n_2) \gets $ Reachable($n-k+1$, $n$, $g_1$)
                \If{$r_2$}
                    \State $(r_3, i_3, n_3) \gets $ Reachable($n+1$, $n_2 + k$, $g_1$)
                    \State $n_p \gets $ Min($n_p$, $n_3$)
                    \State $F \gets$ $F$ $\cup $ ($\neg counterExample$, $counterExample$)
                    \State $G \gets$ $G$ $\cup $ ($\neg counterExample$, $counterExample$)
                \Else
                    \State $g_3 \gets$ $i_2 \wedge lemma$
                    \State $F \gets$ $F$ $\cup $ ($g_3$, $counterExample$)
                    \State $F \gets$ $F$ $\setminus $ ($lemma$, $counterExample$)
                    \State $Q$.push(($g_3$, $counterExample$))
                \EndIf
                \State \Return ($F$, $G$, $n_p$, $isInvalid$)
            \EndWhile

        \end{algorithmic}
    \end{mdframed}
    \caption{Push procedure}\label{alg:3}
\end{figure}

\newpage
\section{PD-Kind procedure}

\noindent The main PDKind procedure shown in Algorithm \ref{alg:4}
checks if property \( P \) is invariant by
iteratively calling the \( Push \) procedure to find a k-inductive
strengthening of \( P \) for some $1 \leq k \leq n + 1$.
PDKind iteratively strengthens the invariant until it becomes inductive.
The strengthening
\( G \) is k-inductive, and if $F == G$, then no more strengthening is required, and \( P \) is proven to be invariant
,so we return \( SAFE \). If the \( Push \) procedure marks
\( isInvalid \) as \( True \), then a counterexample has been found, proving that the property is not invariant and we
return \( UNSAFE \). Otherwise, we update \( n \) and repeat the loop.


\begin{figure}[H]
\begin{mdframed}
\begin{algorithmic}[1]
\State \textbf{Input:} Initial states $I$, transition formula $T$, property $P$
\State \textbf{Output:} Return UNSAFE if P is invalid or SAFE when there is no inductive strengthening left

\State $n \gets 0$
    \State $F \gets (P, \neg P)$
    \While{true}
        \State $k \gets n + 1$
        \State $(F, G, n_p, isInvalid) \gets$ Push($F$, $n$, $k$)
        \If{$isInvalid$}
        \State \Return UNSAFE
        \EndIf
        \If{$F = G$}
        \State \Return SAFE
        \EndIf
        \State $n \gets n_p$
        \State $F \gets G$
    \EndWhile

\end{algorithmic}
\end{mdframed}
\caption{Main PD-Kind procedure}\label{alg:4}
\end{figure}

