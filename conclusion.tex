\chapter{Conclusion}

In this work, we introduced the concept of the CHC framework. Then, we defined the concepts of transition systems and satisfiability modulo theories. After that, we described the structure of Golem and analyzed the best way to integrate a new engine into it. We then analyzed the entire PDKind algorithm and modified it to serve our needs.

With this knowledge, we implemented the PDKind engine itself. The goal was to create an engine that would (1) be well integrated into the Golem framework, (2) return correct answers, and (3) match the performance of other engines. To validate these goals, we experimentally compared our engine with other existing Golem engines. We managed to achieve results comparable to other engines, and in some cases, our engine was faster, especially on SAT benchmarks. While PDKind did not reach the top performance in total solved instances, it successfully solved several problems that no other engine could, showing its value as a complementary solver in Golem’s engine portfolio.

We believe that this engine presents a well-performing alternative to other engines. Its strengths on certain benchmarks and its unique problem coverage make it a useful addition, and its usefulness is expected to grow with further improvements.
