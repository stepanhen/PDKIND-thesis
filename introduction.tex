\chapter{Introduction}

\noindent In computer science and software engineering, the idea of software verification is becoming more and more important. Verification checks whether the program or system operates correctly and fulfills the set properties. The goal is to detect bugs during the development process.

One of the frameworks that is becoming popular is the Constrained Horn Clauses (CHC) framework. CHC is a fragment of First Order Logic modulo constraints that captures many program verification problems as constraint solving. The main advantage of CHC is that it separates modeling from solving by translating the program's behavior and properties into constrained language and then using a specialized CHC solver to solve various verification tasks across programming languages by deciding the satisfiability problem of a CHC system.

Golem \cite{blicha_golem_2023} is one such solver, which integrates the interpolating SMT solver OpenSMT \cite{10.1007/978-3-319-40970-2_35}. Golem currently implements six model-checking algorithms to solve the CHC satisfiability problem.

On top of solving the CHC satisfiability problem, each engine in Golem provides a validity witness for their answer. In software verification we can think of these witnesses as invariants for SAFE answers and counterexample path for UNSAFE answer. By providing these witnesses, we can ensure that the engines answer is correct. Also to check, wheter the engine doesn't give false witnesses, Golem has built in an internal validator to verify the correctness of the witness. 

Describe PDKIND

\section{Goals}
\noindent Our goal in this work is to 
