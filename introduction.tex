\chapter{Introduction}

\noindent In computer science and software engineering, the idea of software
verification is becoming increasingly important. Software verification is the
process of checking whether a program or system operates correctly and meets
its specification. The goal is to detect bugs and errors during the development
process and ensure reliability, especially in critical systems.

There are several approaches to software verification. Some are based on
interactive theorem proving, where the user manually constructs proofs using
logical frameworks. Others rely on static analysis techniques that
over-approximate a program's behavior to detect potential bugs without
executing the code. 

In contrast, \textbf{model checking} is a fully automated technique that explores the
state space of a system and checks whether the desired properties hold. It is
especially suited for verifying safety properties, which ensure that "nothing
bad ever happens" during execution. These properties are typically expressed as
logical formulas over program states, and the system is modeled as a
\textbf{transition system}, which defines how the system evolves from one state
to another.

One formalism gaining popularity in this area is the Constrained Horn Clauses
(CHC)\cite{10.1007/978-3-031-13185-1_2} framework. CHCs form a fragment of
first-order logic modulo constraints that can express many program verification
problems as constraint-solving tasks. The key advantage of CHCs is that they
separate modeling from solving by translating the program’s behavior and
properties into logical clauses, which can then be passed to a solver to decide
satisfiability.

Golem\cite{blicha_golem_2023} is one such solver that integrates the
interpolating SMT solver OpenSMT\cite{10.1007/978-3-319-40970-2_35}. Golem
currently implements six model-checking algorithms for solving CHC
satisfiability problems.

In addition to solving the satisfiability problem, each engine in Golem
provides a validity witness for its answer. In software verification, these
witnesses can be viewed as inductive invariants for SAFE answers and
counterexample paths for UNSAFE answers. By including a witness, the engine's
answer can be validated. Moreover, to guard against incorrect witnesses, Golem
has a built-in internal validator that checks each witness's correctness.

One such algorithm for solving CHC problems is PDKind
(Property-Directed K-induction)\cite{7886665}. PDKind is an
IC3\cite{6148908}-based algorithm that separates reachability checking from
induction reasoning, allowing the induction step to be replaced with
k-induction.

While IC3 uses simple induction, which may not be sufficient to prove more
complex properties, k-induction can consider multiple steps at once, making it
more powerful in some cases. By combining the structure of IC3 with the
strength of k-induction, PDKind aims to improve the performance of model
checking, especially when verifying difficult safety properties.

\section{Goals}

\noindent The goal of this work is to implement the PDKind algorithm as a new
engine and integrate it into the Golem solver\cite{blicha_golem_2023}.

First, we present the necessary background on SMT solving, model checking, and
CHCs. Then, we introduce the architecture of the Golem solver and analyze the
PDKind algorithm in detail. Based on this analysis, we design a solution for
integrating PDKind as a new engine in Golem.

Next, we implement the design and extend it to generate validity witnesses for
both SAFE and UNSAFE answers. We evaluate the implementation on a set of
benchmarks, comparing its correctness and efficiency with the other existing
engines in Golem.

Finally, we use Golem’s internal validator to verify that the produced
witnesses are correct. Throughout the process, we reflect on the design
decisions and implementation challenges, with the aim of contributing a
reliable and reusable component to the Golem verification ecosystem.

